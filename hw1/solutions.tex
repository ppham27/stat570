\documentclass[letterpaper,11pt]{article}

\usepackage{amsmath}
\usepackage{amssymb}
\usepackage[hmargin=1.25in,vmargin=1in]{geometry}
\usepackage{booktabs}
\usepackage{graphicx}
\usepackage{hyperref}
\usepackage{lmodern}
\usepackage{microtype}

\title{Coursework 1: STAT 570}
\author{Philip Pham}
\date{\today}

\begin{document}
\maketitle

\begin{enumerate}
\item The data we analyze are from a 1970s study that investigated insurance
  redlining on $n = 47$ zipcodes. Information on who was being refused
  homeowners is not available so instead we take as response the number of FAIR
  plan policies written and renewed in Chicago by zip code over the period
  December 1977 to May 1978. The FAIR plan was offered by the city of Chicago as
  a default policy to homeowners who had been rejected by the voluntary market.
  The data we will analyze are named \texttt{chredlin} and are in the
  \texttt{faraway} package. The variable \texttt{involact} are the number of new
  FAIR plan policies and renewals per 100 housing units.

  We will consider five covariates for modeling the response: racial composition
  in percent minority (\texttt{race} $x_{i1}$), fires per 100 housing units
  (\texttt{fire} $x_{i2}$), theft per 1000 population (\texttt{theft} $x_{i3}$),
  percent of housing units built before 1939 (\texttt{age} $x_{i4}$), log
  median family income in thousands of dollars (\texttt{lincome} $x_{i5}$`),
  $i = 1,\ldots,47$.

  We will examine the model with the main effects due to race, fire, theft, age
  and $\log(\mathrm{income})$.

  We let $Y_i$ represent \texttt{involact}, and
  $x_i = \left(x_{i1}, x_{i2}, \ldots, x_{i5}\right)$, the covariates, for
  individual $i$, $i = 1,2,\ldots,47$. We fit the model
  \begin{equation}
    y_i = \beta_0 + \sum_{j=1}^5x_{ij}\beta_j + \epsilon_i
    \label{eqn:p1_model}
  \end{equation}
  for $i=1,\ldots,n$ using least squares.

  \begin{enumerate}
  \item Provide informative plots to illustrate what we might expect to learn
    from the model in Equation \ref{eqn:p1_model}.

    \begin{description}
    \item[Solution:] See Figure \ref{fig:p1_pair_plots} and the corresponding
      code in
      \href{https://nbviewer.jupyter.org/github/ppham27/stat570/blob/master/hw1/chredlin\_explore.ipynb}{\texttt{chredlin\_explore.ipynb}}.

      \texttt{fire}, \texttt{race}, and \texttt{age} appear to be positively
      correlated with \texttt{involact}. \texttt{income} appears to be
      negatively correlated.

      Zipcodes in the northern \texttt{side} of Chicago have a lower minority
      population and higher income. \texttt{involact} is smaller in these
      northern zipcodes, too.
    \end{description}
  \item Give interpretations of the parameters $\beta_j$, $j = 1,\ldots,5$.

    \begin{table}
      \centering
      \begin{tabular}{lrrrr}
\toprule
{} &  estimate &  std\_error &  t-statistic &   p-value \\
\midrule
(intercept) & -1.185540 &   1.100255 &    -1.077514 &  0.287550 \\
race        &  0.009502 &   0.002490 &     3.816831 &  0.000449 \\
fire        &  0.039856 &   0.008766 &     4.546588 &  0.000048 \\
theft       & -0.010295 &   0.002818 &    -3.653264 &  0.000728 \\
age         &  0.008336 &   0.002744 &     3.037749 &  0.004134 \\
log\_income  &  0.345762 &   0.400123 &     0.864137 &  0.392540 \\
\bottomrule
\end{tabular}

      \caption{The result of fitting the model described in Equation
        \ref{eqn:p1_model}. The procedure for obtaining the estimates and test
        statistics is described in Part \ref{part:p1_model}.}
      \label{tab:p1_model_parameters}
    \end{table}          

    \begin{description}
    \item[Solution:] Fitting such a model, we get the estimates in Table
      \ref{tab:p1_model_parameters} for $\beta_j$.

      The percent of minorities (\texttt{race}) and frequency of fires
      (\texttt{fire}) are positively correlated with the number of FAIR plan
      policies. \texttt{involact} is the number of FAIR plans per 100 housing
      units. Thus, every percent increase in racial minorities means about 1
      FAIR plan, and for every fire per 100 housing units, there are 3 FAIR
      plans.

      \texttt{age} seems to have postive effect on \texttt{involact}, while
      \texttt{theft} has a negative effect.

      \texttt{log\_income} doesn't seem to tell us anything new: it's correlated
      with other covariates, and its effect is mainly due to chance.      
    \end{description}
    
  \item Reproduce every number in the handout using matrix and arithmetic
    operations.
    \label{part:p1_model}

    \begin{description}
    \item[Solution:] Let us assume that
      $\epsilon_i \sim \mathcal{N}\left(0, \sigma^2\right)$. The log-likelihood
      of this model is
      \begin{align}
        \sum_{i=1}^n \log\mathbb{P}\left(y_i \mid x_i, \beta, \sigma^2\right)
        &= -\frac{n}{2}\log\left(2\pi\sigma^2\right)
          - \frac{1}{2\sigma^2}\sum_{i=1}^n \left(y_i - x_i^\intercal\beta\right)^2
          \nonumber \\
        &= -\frac{n}{2}\log\left(2\pi\sigma^2\right) - \frac{1}{2\sigma^2}\left\lVert
          y - X\beta
          \right\rVert_2^2,
          \label{eqn:p1_log_likelihood}
      \end{align}
      where we $0$-index $\beta$ and the columns of $X$, so each row of $X$ is
      $x_i = \left(1, x_{i1}, x_{i2},\ldots,x_{i5}\right)$.

      \subsection*{Estimating $\hat{\beta}$}

      To maximize Equation \ref{eqn:p1_log_likelihood}, we choose $\hat{\beta}$
      such that $X\hat{\beta}$ is the projection of $y$ onto the hyperplane
      spanned by the columns of $X$. Thus, we must have that
      $X^\intercal\left(y - X\hat{\beta}\right) = 0$ since the residuals will
      orthogonal to the columns of $X$ if $X\hat{\beta}$ is the projection that
      minimizes the squared error. Solving for $\hat{\beta}$, we have that
      \begin{equation}
        \hat{\beta} = \left(X^\intercal X\right)^{-1} X^\intercal y.
        \label{eqn:p1_beta_hat}
      \end{equation}
      The results of apply Equation \ref{eqn:p1_beta_hat} can be seen in the
      first column of Table \ref{tab:p1_model_parameters}.

      \subsection*{Estimating $\hat{\sigma}^2$}

      Let us derive an unbiased estimator for residual standard error. Consider
      the residual random vector.

      \begin{equation}
        R = y - X\hat{\beta}
      \end{equation}

      As stated earlier, the residuals are orthogonal to hyperplane spanned by
      the columns of $X$, so they must lie in some orthonormal hyperplane of
      $N - p$ vectors, where $p = \dim(\beta)$. Thus, residuals are $y$
      projected down to this space.

      Let $w_1,\ldots,w_{n-p}$ be an orthonormal basis of this space. Let $W$ be
      matrix with these basis vectors as the columns.

      We have that
      \begin{align}
        R &= y - X\hat{\beta} \nonumber\\
          &= W\left(W^\intercal y\right) \nonumber\\
          &= W\left(W^\intercal\left(X\beta + \sigma\epsilon\right)\right) \nonumber\\
          &= W\left(W^\intercal X\right)\beta + \sigma W\left(W^\intercal\epsilon\right) \nonumber\\
          &= \sigma W\left(W^\intercal\epsilon\right).
      \end{align}

      Now, $W^\intercal\epsilon \sim \mathcal{N}\left(0, I_{n-p}\right)$. To see
      this, note that the $i$th entry is
      $\sum_{j=1}^n w_{ij}\epsilon_j \sim \mathcal{N}\left(0, 1\right)$, and for
      $i \neq i^\prime$,
      \begin{align*}
        \operatorname{Cov}\left(
        \left(W^\intercal\epsilon\right)_i, \left(W^\intercal\epsilon\right)_{i^\prime}\right)
        &=
          \mathbb{E}\left[
          \left(\sum_{j=1}w_{ij}\epsilon_j\right)\left(\sum_{k=1}w_{i^\prime k}\epsilon_k\right)
          \right] \\
        &= \sum_{j=1}^n\mathbb{E}\left[w_{ij}w_{i^\prime j} \epsilon_j^2\right] +
          2\sum_{j=1}^{n-1}\sum_{k=j+1}^n \mathbb{E}\left[w_{ij}w_{i^\prime k} \epsilon_j\epsilon_k\right] \\
        &= w_i^\intercal w_{i^\prime} + 2\sum_{j=1}^{n-1}\sum_{k=j+1}^n w_{ij}w_{i^\prime k}
          \mathbb{E}\left[\epsilon_j\epsilon_k\right] \\
        &= 0,
      \end{align*}
      where the first term disappears by since the two vectors are orthonormal,
      and the second term disappears because of independence of the errors.

      Thus, we have that

      \begin{equation}
        R^\intercal R
        = \sigma^2 \left(W^\intercal\epsilon\right)^\intercal W^\intercal W \left(W^\intercal\epsilon\right)
        = \sigma^2 \left(W^\intercal\epsilon\right)^\intercal\left(W^\intercal\epsilon\right)
        \sim \sigma^2 \chi^2_{n-p}.
        \label{eqn:p1_residual_distribution}
      \end{equation}

      Finally, we have that

      \begin{equation*}
        \mathbb{E}\left[R^\intercal R\right] = \sigma^2\left(n - p\right)
        \Rightarrow
        \mathbb{E}\left[\frac{\sum_{i=1}^n \left(y - X\hat{\beta}\right)^2}{n-p}\right] = \sigma^2.
      \end{equation*}

      Our consistent estimator is

      \begin{equation}
        \hat{\sigma}^2 = \frac{\sum_{i=1}^n \left(y - X\hat{\beta}\right)^2}{n-p}.
        \label{eqn:p1_sample_variance}
      \end{equation}

      Applying Equation \ref{eqn:p1_sample_variance}, we obtain
      \boxed{\hat{\sigma} = \input{p1_residual_standard_error.txt}\unskip.}

      \subsection*{Hypothesis Testing}

      We can rewrite $y$ as $y = X\beta + \sigma \epsilon$, where each element
      of $\epsilon$ is drawn from $\mathcal{N}\left(0, 1\right)$. Substituting,
      we have that
      \begin{align}
        \hat{\beta}
        &= \left(X^\intercal X\right)^{-1}X^\intercal\left(X\beta + \sigma\epsilon\right) \nonumber\\
        &= \beta + \sigma\left(X^\intercal X\right)^{-1}X^\intercal \epsilon.
          \label{eqn:p1_beta_hat_distribution}
      \end{align}

      Thus, $\hat{\beta}_j \sim \mathcal{N}\left(\beta_j, \sigma^2\left(X^\intercal X\right)^{-1}_{jj}\right)$.

      This gives us that
      \begin{equation*}
        \frac{\hat{\beta}_j - \beta_j}{\sqrt{\sigma^2\left(X^\intercal X\right)^{-1}_{jj}}} \sim
        \mathcal{N}\left(0, 1\right).
      \end{equation*}

      From Equations \ref{eqn:p1_residual_distribution} and \ref{eqn:p1_sample_variance},
      \begin{equation}
        (n - p)\frac{\hat{\sigma}^2}{\sigma^2} \sim \chi^2_{n-p}.
      \end{equation}

      $\hat{\beta}$ and $\hat{\sigma}^2$ are independent by
      \href{https://en.wikipedia.org/wiki/Basu\%27s_theorem}{Basu's theorem}:
      $\hat{\sigma}^2$ is an ancillary statistic that does not depend on the
      model parameters, $\beta$. Thus, we have that
      \begin{equation}
        \left.
          \frac{\hat{\beta}_j - \beta_j}{\sqrt{\sigma^2\left(X^\intercal X\right)^{-1}_{jj}}}
          \middle/
          \sqrt{\frac{(n - p)\frac{\hat{\sigma}^2}{\sigma^2}}{n-p}}
        \right. 
        = \frac{\hat{\beta}_j - \beta_j}{\sqrt{\hat{\sigma}^2\left(X^\intercal X\right)^{-1}_{jj}}}
        \sim t_{n-p}.
        \label{eqn:p1_beta_hat_j_distribution}
      \end{equation}

      That is, we have $t$ distribution with $n - p$ degrees of freedom. The
      denominator of Equation \ref{eqn:p1_beta_hat_j_distribution} gives the
      second column of Table \ref{tab:p1_model_parameters}.

      For each $\beta_j$, our null hypothesis is $H_0: \beta_j = 0$. Thus, our
      $t$-test statistic is obtain from substituting $\beta_j$ into Equation
      \ref{eqn:p1_beta_hat_j_distribution},
      \begin{equation*}
        \hat{t}_j = \frac{\hat{\beta}_j}{\sqrt{\hat{\sigma}^2\left(X^\intercal X\right)^{-1}_{jj}}},
      \end{equation*}
      which gives us the third column of Table \ref{tab:p1_model_parameters}.

      The fourth column is the probability of obtaining evidence that
      contradicts the null hypothesis at least as much. Let $F^{-1}_{t_{n-p}}$
      be the inverse cumulative distribution function. The $p$-value is
      \begin{equation*}
        \mathbb{P}\left(
          \left\lvert T_{n - p}\right\rvert \geq
          \left\lvert \hat{t}_j\right\rvert
          \mid
          \hat{t}_j
        \right) = 
        2\left(1 - F^{-1}_{n - p}\left(\left\lvert\hat{t}_j\right\rvert\right)\right).
      \end{equation*}
    \end{description}
  \end{enumerate}
\end{enumerate}

\begin{figure}
  \centering
  \includegraphics[width=\textwidth]{p1_pair_plots.pdf}
  \caption{The empirical univariate and joint distributions for the
    \texttt{chredlin} dataset.}
  \label{fig:p1_pair_plots}
\end{figure}

\end{document}

