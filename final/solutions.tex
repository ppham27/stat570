\documentclass[letterpaper,11pt]{article}

\usepackage{amsmath}
\usepackage{amssymb}
\usepackage{bm}
\usepackage[hmargin=1.25in,vmargin=1in]{geometry}
\usepackage{booktabs}
\usepackage{graphicx}
\usepackage{hyperref}
\usepackage{lmodern}
\usepackage{microtype}
\usepackage{subcaption}

\title{Final: STAT 570}
\author{Philip Pham}
\date{\today}

\begin{document}
\maketitle

Consider the failure time data in Table \ref{tab:failure_time_data}.

\begin{enumerate}
\item We describe a simple model for these data. Let $p$ ($0 < p < 1$) denote
  the weekly failure probability, i.e., the probability of failure during any
  week, and $T$ the random variable describing the week at which failure
  occurred. Then $T$ may be modeled as a geometric random variable:
  \begin{equation}
    \mathbb{P}\left(T = t \mid p\right)
    = \begin{cases}
      p\left(1-p\right)^{t-1}, &t=1,2,\ldots; \\
      0,&\text{otherwise}.      
    \end{cases}
    \label{eqn:p1_model}
  \end{equation}

  Let $Y_t$ represent the number of components that fail in week $t$,
  $t = 1,2,\ldots,N$, and $Y_{N+1}$ the number of components that have not failed
  by week $N$.

  \begin{enumerate}
  \item Show that the likelihood function is
    \begin{equation}
      L\left(p\right) =
      \left[\left(1 - p\right)^N\right]^{Y_{N+1}}
      \prod_{t=1}^N\left[
        p\left(1 - p\right)^{t-1}
      \right]^{Y_t}.
      \label{eqn:p1_likelihood}
    \end{equation}
    \begin{description}
    \item[Solution:] An individual component's failure week has distribution
      $\operatorname{Geometric}\left(p\right)$. The probability that a single
      component fails in week $t$ is the probability that it survived $t - 1$
      weeks and failed on week $t$, which is $p\left(1 - p\right)^{t-1}$ from
      Equation \ref{eqn:p1_model}. There are $Y_t$ such components, which gives
      us the factors for $t = 1,2,\ldots,N$.

      The probability that a component fails at a later date is
      \[
        \left(1 - p\right)^N\sum_{k=1}^\infty p\left(1-p\right)^{k-1}
        =
        \left(1 - p\right)^N\frac{p}{1 - \left(1 - p\right)}
        =
        \left(1 - p\right)^N,
      \]
      which gives us the remaining factor. There are $Y_{N+1}$ remaining
      components, so
      \[
        L\left(p\right) =
        \left\{
          \prod_{t=1}^N \left[
            p\left(1 - p \right)^{t-1}
          \right]^{Y_t}
        \right\}
        \times
        \left[\left(1-p\right)^N\right]^{Y_{N+1}}.
      \]
    \end{description}
  \item Find an expression for the MLE $\hat{p}$.
    \begin{description}
    \item[Solution:] 
      The score function is
      \begin{align}
        S\left(p\right)
        &= \frac{\partial}{\partial p}\log L\left(p\right) \nonumber\\
        &= \frac{\partial}{\partial p}\left[
          NY_{N + 1}\log\left(1 - p\right)
          + \sum_{t=1}^N Y_t\left(
          \log p + \left(t - 1\right)
          \log\left(1 - p\right)
          \right)
          \right]\nonumber\\
        &= -\frac{NY_{N+1}}{1 - p}
          + \sum_{t=1}^N Y_t\left(
          \frac{1}{p}
          -
          \frac{t - 1}{1 - p}
          \right)
        = -\frac{NY_{N+1}}{1 - p}
          + \sum_{t=1}^N Y_t
          \frac{1-pt}{p\left(1 - p\right)}.
          \label{eqn:p1_score}
      \end{align}

      Solving for $S\left(\hat{p}\right) = 0$, we find the MLE:
      \begin{equation}
        \hat{p}\left(
          NY_{N+1} + \sum_{t=1}^{N}tY_t
        \right) = \sum_{t=1}^N Y_t
        \implies
        \boxed{
          \hat{p} = \frac{\sum_{t=1}^N Y_t}
          {NY_{N+1} + \sum_{t=1}^{N}tY_t}.
        }
        \label{eqn:p1_mle}
      \end{equation}
    \end{description}
  \item Find the form of the observed information and hence the asymptotic
    variance of the maximum likelihood estimate (MLE).
    \begin{description}
    \item[Solution:] Using Equation \ref{eqn:p1_score}, the expected observed
      information is
      \begin{align}
        I\left(p\right)
        &= \mathbb{E}\left[-\frac{\partial}{\partial p}S\left(p\right) \mid p\right]
          \nonumber\\
        &= \frac{N\mathbb{E}\left[Y_{N+1} \mid p\right]}{(1-p)^2}
          + \sum_{t=1}^N\mathbb{E}\left[Y_t \mid p\right]
          \left(
          \frac{1}{p^2} + \frac{t-1}{\left(1-p\right)^2}
          \right)
          \nonumber \\
        &= n\frac{N\left(1-p\right)^N}{(1-p)^2}
          + np\sum_{t=1}^N
          \left(1 - p\right)^{t-1}
          \left(
          \frac{1}{p^2} + \frac{t-1}{\left(1-p\right)^2}
          \right)
          \nonumber \\
        &= n\left[
          \frac{\left(1-p\right)^N}{(1-p)^2}
          +
          \frac{1 - \left(1 - p\right)^N}{p^2}
          +
          \frac{(1-p) - (1-p)^N}{p(1-p)^2}
          \right] \nonumber \\
        &= \boxed{n\frac{1 - \left(1 - p\right)^N}
          {p^2\left(1 - p\right)},} \label{eqn:p1_fisher_information}
      \end{align}
      where $n = Y_{N+1} + \sum_{t=1}^N Y_t$.

      From Equation \ref{eqn:p1_fisher_information}, the asymptotic variance of
      $\hat{p}$ is
      \begin{equation}
        \operatorname{var}\left(\hat{p}\right)
        \approx
        \hat{\operatorname{var}}\left(\hat{p}\right) = 
        I\left(\hat{p}\right)^{-1} = \frac{1}{n} \times
        \frac{\hat{p}^2\left(1-\hat{p}\right)}
        {1 - \left(1-\hat{p}\right)^N}
        \label{eqn:p1_variance}
      \end{equation}
      by asymptotic normality of the MLE.
    \end{description}
  \item For the data in Table \ref{tab:failure_time_data}, calculate the MLE,
    $\hat{p}$. the variance of $\hat{p}$, and an asymptotic
    95\% confidence interval for $p$.

    \begin{description}
    \item[Solution:] The MLE can be calculated with Equation \ref{eqn:p1_mle} to
      be $\boxed{\hat{p} = 0.354717.}$ The variance can be found with Equation
      \ref{eqn:p1_variance} to be
      $\boxed{\hat{\operatorname{var}}\left(\hat{p}\right) = 0.00016828.}$

      If $\Phi$ is the cumulative distribution function for a standard normal,
      we can use asymptotic normality to find the 95\% confidence interval as
      \begin{equation*}
        \left[
          \hat{p} +
          \Phi^{-1}\left(0.025\right)\sqrt{\hat{\operatorname{var}}\left(\hat{p}\right)},
          \hat{p} +
          \Phi^{-1}\left(0.975\right)\sqrt{\hat{\operatorname{var}}\left(\hat{p}\right)}
        \right]
        =
        \boxed{\left[0.32929,0.38014\right].}
      \end{equation*}
    \end{description}
  \item We now consider a Bayesian analysis. The conjugate prior for $p$ is a
    beta distribution, $\operatorname{Beta}\left(a, b\right)$. State the form of
    the posterior with this choice. Give the form of the posterior mean and
    write as a weighted combination of the MLE and the prior mean.

    \begin{description}
    \item[Solution:] By Bayes' rule, we know the posterior density is
      proportional to the likelihood times the prior. From Equation
      \ref{eqn:p1_likelihood}, we'll have
      \begin{align*}
        L\left(p\right) \times \left[
        p^{a-1}\left(1-p\right)^{b-1}
        \right]
        &= p^{a-1}\left(1 - p\right)^{b + NY_{N+1} - 1}
          \prod_{t=1}^N\left[
          p\left(1 - p\right)^{t-1}
          \right]^{Y_t} \\
        &= p^{a + \sum_{t=1}^N Y_t -1}
          \left(1 - p\right)^{b + \sum_{t=1}^N (t-1)Y_t + NY_{N+1} - 1},
      \end{align*}
      whose form we recognize as the integrand of beta function, so the
      posterior also has beta distribution, that is,
      \begin{align}
        p \mid Y_1,Y_2,\ldots,Y_{N+1}
        &\sim \operatorname{Beta}\left(
          a + \sum_{t=1}^N Y_t,
          b + \sum_{t=1}^N (t-1)Y_t + NY_{N+1}
          \right) \nonumber\\
        &= \frac{\Gamma\left(a^\prime + b^\prime\right)}
          {\Gamma\left(a^\prime\right)\Gamma\left(b^\prime\right)}
          p^{a^\prime -1}\left(1 - p\right)^{b^\prime - 1},
          \label{eqn:p1_posterior}
      \end{align}
      where $a^\prime = a + \sum_{t=1}^N Y_t$ and
      $b^\prime = b + \sum_{t=1}^N (t-1)Y_t + NY_{N+1}$.

      The posterior mean takes the form
      \begin{align}
        \mathbb{E}\left[
        p \mid Y_1,Y_2,\ldots,Y_{N+1}
        \right]
        &= \frac{a^\prime}{a^\prime + b^\prime}
          \nonumber \\
        &= \frac{a + \sum_{t=1}^N Y_t}
          {a + b + \sum_{t=1}^N tY_t + NY_{N+1}}. \label{eqn:p1_posterior_mean}
      \end{align}

      We have that the prior mean is $p_{\mathrm{prior}} = \frac{a}{a + b}$.
      Equation \ref{eqn:p1_posterior_mean} can be rewritten as
      \begin{equation}
        \boxed{
        \frac{
          \left(a + b\right)
          p_{\mathrm{prior}}
          +
          \left(
            \sum_{t=}^N tY_t + NY_{N+1}
          \right)
          \hat{p}}
          {a + b + \sum_{t=}^N tY_t + NY_{N+1}},}
        \label{eqn:p1_posterior_mean_sum}
      \end{equation}
      so the posterior mean is a convex combination of the prior mean and MLE.
    \end{description}
  \item Suppose we wish to fix the parameters of the prior, $a$ and $b$, so that
    the mean is $\mu$ and the prior standard deviation is $\sigma$. Obtain
    expressions for $a$ and $b$ in terms of $\mu$ and $\sigma^2$.
    \begin{description}
    \item[Solution:] It is well known that the mean and variance of the
      $\operatorname{Beta}\left(a,b\right)$ distribution are $\frac{a}{a+b}$ and
      $\frac{ab}{(a+b)^2(a+b+1)}$, respectively.

      Solving equations
      \begin{align*}
        \frac{a}{a + b}
        &= \mu \\
        \frac{ab}{(a+b)^2(a+b+1)} &= \sigma^2,
      \end{align*}
      we find that
      \begin{align}
        a
        &= \mu\left[
          \frac{\mu\left(1 - \mu\right)}{\sigma^2} - 1
          \right]
          \label{eqn:p1_a} \\
        b
        &=  \left(1 - \mu\right)\left[
          \frac{\mu\left(1 - \mu\right)}{\sigma^2} - 1
          \right].
          \label{eqn:p1_b}
      \end{align}
    \end{description}
  \item For the data in Table \ref{tab:failure_time_data}, assume we wish to
    have a beta prior with $\mu = 0.2$ and $\sigma = 0.08$. State the posterior
    for the prior corresponding to this choice and evaluate the posterior
    mean. Simulate samples from the posterior distribution. Provide a histogram
    representation of the posterior distribution and calculate the 5\%, 50\% and
    95\% points of the posterior distribution.

    \begin{figure}
      \centering
      \includegraphics{p1_posterior_samples.pdf}
      \caption{2,048 samples drawn from the posterior in Equation
        \ref{eqn:p1_posterior_data}. The red ticks denote the 5\%, 50\% and
        95\% quantiles.}
      \label{fig:p1_posterior_samples}
    \end{figure}
    
    \begin{description}
    \item[Solution:] Apply Equations \ref{eqn:p1_a} and \ref{eqn:p1_b} with
      $\mu = 0.2$ and $\sigma = 0.08$, we find the prior:
      \begin{equation}
        p \sim \operatorname{Beta}\left(4.8, 19.2\right).
        \label{eqn:p1_prior}
      \end{equation}

      Using Equation \ref{eqn:p1_posterior}, we have find the posterior:
      \begin{equation}
        p \sim \operatorname{Beta}\left(474.8, 874.2\right).
        \label{eqn:p1_posterior_data}
      \end{equation}
      
      A histogram of samples drawn from the distribution in Equation
      \ref{eqn:p1_posterior_data} can be found in Figure
      \ref{fig:p1_posterior_samples}. The 5\%, 50\%, and 95\% posterior
      quantiles are 0.33070873, 0.35189124, and 0.37346975, respectively.
    \end{description}
  \end{enumerate}
\item
  \begin{enumerate}
  \item A more complex likelihood for these data would assume that the $i$-th
    component had their own probability $p_i$, with the $p_i$'s arising from a
    distribution $\pi\left(p\right)$. Show that
      \begin{equation}
        \mathbb{P}\left(T = t\right) =
        \mathbb{E}\left[(1 - p)^{t-1}\right] -
        E[(1 - p)^t],
        \label{eqn:p2_pmf}
      \end{equation}
      and
      \begin{equation}
        \mathbb{P}\left(T > t\right) = \mathbb{E}\left[(1 - p)^t\right].
        \label{eqn:p2_survival}
      \end{equation}
      \begin{description}
      \item[Solution:] First let us find the survival function in
        \ref{eqn:p2_survival}.
        \begin{align*}
          \mathbb{P}\left(T > t\right)
          &= \int_0^1
          \mathbb{P}\left(T > t \mid p\right)
          \pi\left(p\right)\,\mathrm{d}p 
          = \int_0^1 \left[\sum_{s=t + 1}^\infty p(1 - p)^{s-1}\right]
            \pi\left(p\right)
          \,\mathrm{d}p \\
          &= \int_0^1 \left[p\sum_{s=0}^\infty (1 - p)^s\right]
            (1 - p)^t
            \pi\left(p\right)
            \,\mathrm{d}p \\
          &= \int_0^1 \left[p \times \frac{1}{1 - (1-p)}\right]
            (1 - p)^t
            \pi\left(p\right)
            \,\mathrm{d}p
          = \int_0^1 (1 - p)^t
            \pi\left(p\right)
            \,\mathrm{d}p \\
          &= \mathbb{E}\left[\left(1-p\right)^t\right],
        \end{align*}
        which proves Equation \ref{eqn:p2_survival}.

        The probability mass function in Equation \ref{eqn:p2_pmf} follows:
        \begin{equation*}
          \mathbb{P}\left(T = t\right)
          = \mathbb{P}\left(T > t - 1\right)
          - \mathbb{P}\left(T > t\right)
          = \mathbb{E}\left[\left(1-p\right)^{t-1}\right]
          - \mathbb{E}\left[\left(1-p\right)^{t}\right].
        \end{equation*}
      \end{description}
    \item Obtain expressions for
      $\mathbb{P}\left(T = t \mid \alpha, \beta\right)$ and
      $\mathbb{P}\left(T > t \mid \alpha, \beta\right)$ with
      $\pi\left(\cdot\right)$ taken as the beta distribution,
      $\operatorname{Beta}\left(\alpha, \beta\right)$.

      \begin{description}
      \item[Solution:] These follow from Equations
        \ref{eqn:p2_pmf} and \ref{eqn:p2_survival}.

        \begin{align}
          \mathbb{P}\left(T > t\right)          
          &= \mathbb{E}\left[
            (1 - p)^t
            \right]
            = \sum_{s=t}^\infty
            \mathbb{E}\left[
            p\left(1 - p\right)^s
            \right] \label{eqn:p2_survival_beta}\\
          &=
            \sum_{s=t}^\infty
            \int_0^p
            p\left(1-p\right)^s
            \frac{\Gamma\left(\alpha + \beta\right)}            
            {\Gamma\left(\alpha\right)\Gamma\left(\beta\right)}
            p^{\alpha - 1}\left(1-p\right)^{\beta-1}
            \,\mathrm{d}p
          \nonumber\\
          &= \sum_{s=t}^\infty
            \int_0^p
            \frac{\Gamma\left(\alpha + \beta\right)}
            {\Gamma\left(\alpha\right)\Gamma\left(\beta\right)}
            p^{\alpha + 1 - 1}\left(1-p\right)^{\beta + s -1}
            \,\mathrm{d}p \nonumber\\
          &= \frac{\Gamma\left(\alpha + \beta\right)} 
            {\Gamma\left(\alpha\right)\Gamma\left(\beta\right)}            
            \sum_{s=t}^\infty
            \frac
            {\Gamma\left(\alpha + 1\right)\Gamma\left(\beta + s\right)}
            {\Gamma\left(\alpha + \beta + s + 1\right)}
            = \alpha\frac{\Gamma\left(\alpha + \beta\right)} 
            {\Gamma\left(\beta\right)}
            \sum_{s=t}^\infty
            \frac
            {\Gamma\left(\beta + s\right)}
            {\Gamma\left(\alpha + \beta + s + 1\right)}
          \nonumber\\
          &= 
            1
            - \alpha
            \frac{\Gamma\left(\alpha + \beta\right)} 
            {\Gamma\left(\beta\right)}
            \sum_{s=0}^{t-1}
            \frac
            {\Gamma\left(\beta + s\right)}
            {\Gamma\left(\alpha + \beta + s + 1\right)} \nonumber\\
          &= 1 - \frac{1}{B\left(\alpha,\beta\right)}
            \sum_{s=0}^{t-1} B\left(\alpha + 1, \beta + s\right)
            = \frac{B\left(\alpha,\beta + t\right)}{B\left(\alpha,\beta\right)}
            = \frac{\Gamma\left(\alpha + \beta\right)\Gamma\left(\beta + t\right)}
            {\Gamma\left(\beta\right)\Gamma\left(\alpha + \beta + t\right)}
            \nonumber
        \end{align}
        where $B$ is the beta function, and we know
        $\mathbb{P}\left(T > 0\right) = 1$.

        Plugging Equation \ref{eqn:p2_survival_beta} into Equation
        \ref{eqn:p2_pmf}, one obtains
        \begin{equation}
          \mathbb{P}\left(T = t\right)
          = \frac{B\left(\alpha + 1, \beta + t - 1\right)}{B\left(\alpha,\beta\right)}
          = \alpha
          \frac{\Gamma\left(\alpha + \beta\right)\Gamma\left(\beta + t - 1\right)}
          {\Gamma\left(\beta\right)\Gamma\left(\alpha + \beta + t\right)}
          \label{eqn:p2_pmf_beta}
        \end{equation}
        for $t \in \mathbb{N}$.
      \end{description}
    \item Using the previous part, write down the likelihood function
      $L\left(\alpha, \beta\right)$ corresponding to data
      $\left\{Y_t\right\}_{t=1}^{N+1}$.

      \begin{description}
      \item[Solution:] Our model for $T$ is different, so we can substitute
        Equations \ref{eqn:p2_pmf_beta} and \ref{eqn:p2_survival_beta} into
        Equation \ref{eqn:p1_likelihood}: we'll have
        $\mathbb{P}\left(T = t\right)$ in place of $p\left(1 - p\right)^{t-1}$
        and $\mathbb{P}\left(T > N\right)$ in place of $\left(1 - p\right)^N$.
        \begin{align}
          L\left(\alpha,\beta\right)
          &= \left[\mathbb{P}\left(T > N\right)\right]^{Y_{N+1}}
            \prod_{t=1}^N\left[\mathbb{P}\left(T = t\right)\right]^{Y_t}
            \label{eqn:p2_likelihood} \\
          &= \left[
            \frac{B\left(\alpha, \beta + N\right)}{B\left(\alpha, \beta\right)}
            \right]^{Y_{N+1}}
            \prod_{t=1}^N \left[
            \frac{B\left(\alpha + 1, \beta + t - 1\right)}
            {B\left(\alpha, \beta\right)}
            \right]^{Y_t}.
            \nonumber
        \end{align}
      \end{description}
    \item Find the MLEs  $\hat\alpha$ and $\hat\beta$ for the data of Table
      \ref{tab:failure_time_data}.
      \begin{description}
      \item[Solution:] From Equation \ref{eqn:p2_likelihood}, we can consider
        the log-likelihood function:
        \begin{align}
          &l\left(\alpha,\beta\right)
          =
          \log L\left(\alpha,\beta\right)
          \label{eqn:p2_log_likelihood} \\
          &= -n\log B\left(\alpha,\beta\right)
            +
            Y_{N+1}\log B\left(\alpha,\beta+N\right)
            +
            \sum_{t=1}^N Y_t\log 
            B\left(\alpha + 1, \beta + t - 1 \right).
            \nonumber
        \end{align}

        The score function is
        \begin{align}
          S\left(\alpha,\beta\right) &= 
          \nabla l\left(\alpha,\beta\right)
          = \begin{pmatrix}
            S_\alpha\left(\alpha,\beta\right) \\
            S_\beta\left(\alpha,\beta\right)
          \end{pmatrix} \label{eqn:p2_score}\\
          S_\alpha\left(\alpha,\beta\right)
          &= -n\left[
            \psi\left(\alpha\right) -
            \psi\left(\alpha + \beta\right)
            \right]
            + Y_{N+1}\left[
            \psi\left(\alpha\right)-
            \psi\left(\alpha + \beta + N\right)              
            \right] \nonumber\\
          &~~~~~+ \sum_{t=1}^N Y_{t}\left[
            \psi\left(\alpha + 1\right) -
            \psi\left(\alpha + \beta + t\right)            
            \right],
            \nonumber \\
          S_\beta\left(\alpha,\beta\right)
          &= -n\left[
            \psi\left(\beta\right) -
            \psi\left(\alpha + \beta\right)
            \right]
            + Y_{N+1}\left[
            \psi\left(\beta + N\right)-
            \psi\left(\alpha + \beta + N\right)              
            \right] \nonumber\\
          &~~~~~+ \sum_{t=1}^N Y_{t}\left[
            \psi\left(\beta + t - 1\right) -
            \psi\left(\alpha + \beta + t\right)            
            \right], \nonumber
        \end{align}p
        where
        $\psi\left(x\right) = \Gamma^\prime\left(x\right)/\Gamma\left(x\right)$
        is the digamma function.

        Numerically solving Equation \ref{eqn:p2_score} for
        $S\left(\hat\alpha,\hat\beta\right) = \mathbf{0}$, I obtain
        $\boxed{\hat\alpha = 1.413336}$ and $\boxed{\hat\beta = 1.38001102}$
        for the MLEs.
      \end{description}
    \end{enumerate}
  \item \begin{enumerate}
    \item Show that the likelihood in Equation \ref{eqn:p1_likelihood} can be
      written as a product of binomial distributions.

      \begin{description}
      \item[Solution:] We can model the data as taking $N$ draws from a binomial
        distribution. Following each draw, we discard the failures and make
        another draw if $t < N$:
        \begin{align}
          L\left(p\right)
          &= \prod_{t=1}^N \left[
            {n - \sum_{s=1}^{t-1}Y_s
            \choose Y_t}p^{Y_t}\left(1 - p\right)^{n - \sum_{s=1}^{t}Y_s}
            \right]
            \label{eqn:p3_likelihood} \\
          &= \prod_{t=1}^N \left[
            {\sum_{s = t}^{N+1}Y_s \choose Y_t}
            p^{Y_t}\left(1 - p\right)^{\sum_{s = t + 1}^{N+1} Y_s}\right],
            \nonumber
        \end{align}
        which is equivlaent to Equation \ref{eqn:p1_likelihood} up to a constant
        of proportionality,

        In Equation \ref{eqn:p3_likelihood}, we have a product of binomial
        probability mass functions, where
        $Y_t \mid Y_1,\ldots,Y_{t-1} \sim \operatorname{Binomial}\left(
          n - \sum_{s=1}^{t - 1}Y_s, p \right).$
      \end{description}
    \item Fit the binomial model, and show that the estimate of the probability
      is identical to that under the previous MLE analysis. Obtain a 95\%
      asymptotic confidence interval for $p$.
      \begin{description}
      \item[Solution:] Since Equation \ref{eqn:p3_likelihood} only differs from
        Equation \ref{eqn:p1_likelihood} by a constant of proporitionality, the
        score function is also Equation \ref{eqn:p1_score}. Thus, the MLE is
        same $\hat{p} = 0.354717$.

        The observed information will also be the same. To calculate the
        expected observed information, we can use the law of total expectation
        and strong induction. For the base case
        $\mathbb{E}\left[Y_1\right] = np$. In general,
        $\mathbb{E}\left[Y_t\right] = np\left(1 - p\right)^{t-1}$ for
        $t = 1,2,\ldots,N$. For $t > 1$, we have
        \begin{align}
          \mathbb{E}\left[Y_t\right]
          &=  
            \mathbb{E}\left[\mathbb{E}\left[Y_t \mid Y_1,\ldots,Y_{t-1}\right]\right]
          =
            \mathbb{E}\left[
            p\left(n - \sum_{s=1}^{t-1}Y_s\right)
            \right] \nonumber\\
          &= p\left(n - \sum_{s=1}^{t-1}\mathbb{E}\left[Y_s\right]\right)
          =
            p\left(n - \sum_{s=1}^{t-1} np\left(1 - p\right)^{s-1}\right)
            \nonumber\\
          &= np\left(1 - p\sum_{s=0}^{t-2}\left(1 - p\right)^r\right)
            = np\left(
            1 - p\frac{1 - \left(1 - p\right)^{t - 1}}{p}\right)
          \nonumber\\
          &= np\left(1 - p\right)^{t-1},
            \label{eqn:p3_yt_mean}
        \end{align}
        which is same as it was under the geometric model.
        
        For $Y_{N+1}$, we have
        \begin{align}
          \mathbb{E}\left[Y_{N+1}\right]
          &= \mathbb{E}\left[
            \mathbb{E}\left[Y_{N+1} \mid Y_1,Y_2,\ldots,Y_N\right]\right]
          = \mathbb{E}\left[
          n - \sum_{t=1}^{N} Y_t
          \right]
          \nonumber\\
          &= n - \sum_{t=1}^{N} \mathbb{E}\left[Y_t\right]
            = n - np\sum_{t=1}^N\left(1 - p\right)^{t-1} \nonumber\\
          &= n - np \frac{1 - \left(1 - p\right)^N}{p} = \left(1 - p\right)^N,
            \label{eqn:p3_yn1_mean}
        \end{align}
        which is also the same as under the geometric model. Therefore, the
        expected observed information is the same as Equation
        \ref{eqn:p1_fisher_information}.

        Then, the asymptotic 95\% confidence interval for $p$ is also
        $\left[0.32929,0.38014\right]$.
      \end{description}
    \item Obtain Pearson residuals and comment on the fit of the model, using
      any plots you feel are appropriate.
      \begin{description}
      \item[Solution:] 
      \end{description}
    \item Fit a binomial model you feel is appropriate.
      \begin{description}
      \item[Solution:] 
      \end{description}
    \end{enumerate}
  \end{enumerate}

\begin{table}
  \centering
  \begin{tabular}{lrr}
\toprule
Time (weeks), $i$ &  Failures, $y_i$ &  Temperature, $x_i$ \\
\midrule
              $1$ &              210 &                24.0 \\
              $2$ &              108 &                26.0 \\
              $3$ &               58 &                24.0 \\
              $4$ &               40 &                26.0 \\
              $5$ &               17 &                25.0 \\
              $6$ &               10 &                22.0 \\
              $7$ &                7 &                23.0 \\
              $8$ &                6 &                20.0 \\
              $9$ &                5 &                21.0 \\
             $10$ &                4 &                18.0 \\
             $11$ &                2 &                17.0 \\
             $12$ &                3 &                20.0 \\
            $>12$ &               15 &                     \\
\bottomrule
\end{tabular}

  \caption{Time until failure for $n = 485$ components, along with average weekly
    temperature.}
  \label{tab:failure_time_data}
\end{table}
\end{document}
