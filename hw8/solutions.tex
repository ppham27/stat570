\documentclass[letterpaper,11pt]{article}

\usepackage{amsmath}
\usepackage{amssymb}
\usepackage{bm}
\usepackage[hmargin=1.25in,vmargin=1in]{geometry}
\usepackage{booktabs}
\usepackage{graphicx}
\usepackage{hyperref}
\usepackage{lmodern}
\usepackage{microtype}
\usepackage{pdflscape}
\usepackage{subcaption}

\title{Coursework 8: STAT 570}
\author{Philip Pham}
\date{\today}

\begin{document}
\maketitle
\begin{enumerate}
\item Consider $n$ experiments with $Z_{ij}$ , $j = 1,2,\ldots, N_i$, the binary
  outcomes within cluster (experiment) $i$ with $Y_i = \sum^N_{j=1} Z_{ij}$,
  $i = 1,\ldots,n$.
  \begin{enumerate}
  \item By writing
    \begin{equation}
      \operatorname{var}\left(Y_i\right)
      = \sum_{j=1}^{N_i} \operatorname{var}\left(Z_{ij}\right)
      + \sum_{j=1}^{N_i}\sum_{j \neq k} \operatorname{cov}\left(Z_{ij}, Z_{ik}\right),
      \label{eqn:p1_var_yi}
    \end{equation}
    show that
    \begin{equation}
      \operatorname{var}\left(Y_i\right)
      = N_ip_i\left(1 - p_i\right) \times
      \left[1 + \left(N_i - 1\right)\tau_i^2\right],
      \label{eqn:p1_var_yi_result}
    \end{equation}
    where $p_i = \mathbb{E}\left[Z_{ij}\right]$ and $\tau_i^2$ is the
    correlation of outcomes within cluster $i$.

    \begin{description}
    \item[Solution:] Using the variance for a Bernoulli random variable and the
      definition of the correlation coefficient, we have that
      \begin{align}
        \operatorname{var}\left(Z_{ij}\right)
        &= p_i\left(1 - p_i\right) \label{eqn:p1_var_cov}\\
        \operatorname{cov}\left(Z_{ij}, Z_{ik}\right)
        &= \tau_i^2p_i\left(1 - p_i\right)~\text{for}~j \neq k.
          \nonumber
      \end{align}
      
      Since $Z_{ij}$ are identically distributed for different $j$, we can
      rewrite Equation \ref{eqn:p1_var_yi} as
      \begin{equation}
        \operatorname{var}\left(Y_i\right)
        = N_i\operatorname{var}\left(Z_{i1}\right) +
        N_i\left(N_i - 1\right)\operatorname{cov}\left(Z_{i1},Z_{i2}\right).
        \label{eqn:p1_var_yi_simplified} 
      \end{equation}

      Applying Equation \ref{eqn:p1_var_cov} to Equation
      \ref{eqn:p1_var_yi_simplified}, we have the result
      \begin{align*}
        \operatorname{var}\left(Y_i\right)
        &= N_ip_i\left(1-p_i\right)
          + N_i\left(N_i - 1\right) \tau_i^2p_i\left(1-p_i\right) \\
        &= N_ip_i\left(1-p_i\right) \times \left[
          1 + \left(N_i - 1\right) \tau_i^2
          \right]
      \end{align*}
      as desired.
    \end{description}
  \item Consider the model
    \begin{align}
      Y_i \mid q_i &\sim \operatorname{Binomial}\left(N_i,q_i\right) \\
      q_i &\sim \operatorname{Beta}\left(a_i,b_i\right),
            \label{eqn:p1_model}
    \end{align}
    where we can parameterize as $a_i = dp_i$, $b_i = d\left(1-p_i\right)$, so
    that
    \begin{align}
      \mathbb{E}\left[q_i\right]
      &= p_i = \frac{a_i}{d} \\
      \operatorname{var}\left(q_i\right)
      &= \frac{p_1\left(1 - p_i\right)}{d+1}.
    \end{align}
    Obtain the marginal moments and show that the variance is of the form in
    Equation \ref{eqn:p1_var_yi_result}, and identify $\tau_i^2$.

    \begin{description}
    \item[Solution:] We have that
      \begin{align}
        \mathbb{P}\left(Y_i = y\right)
        &= \int_0^1 \mathbb{P}\left(Y_i \mid q_i\right)p\left(q_i\right)
          \,\mathrm{d}q_i \nonumber\\
        &= \int_0^1 \left(
          {N_i \choose y }q_i^y \left(1 - q_i\right)^{N_i - y}
          \right)
          \left(
          \frac{\Gamma\left(a_i + b_i\right)}
          {\Gamma\left(a_i\right)\Gamma\left(b_i\right)}
          q_i^{a_i - 1}\left(1 - q_i\right)^{b_i - 1}
          \right)
          \,\mathrm{d}q_i \nonumber\\
        &= {N_i \choose y}
          \frac{\Gamma\left(a_i + b_i\right)}
          {\Gamma\left(a_i\right)\Gamma\left(b_i\right)}
          \int_0^1 
          q_i^{a_i + y - 1} \left(1 - q_i\right)^{b_i + N_i - y - 1}
          \,\mathrm{d}q_i \nonumber\\
        &= {N_i \choose y}
          \left(
          \frac{\Gamma\left(a_i + b_i\right)}
          {\Gamma\left(a_i\right)\Gamma\left(b_i\right)}
          \right)
          \left(
          \frac{\Gamma\left(y + a_i\right)\Gamma\left(
          N_i - y + b_i
          \right)}
          {\Gamma\left(N_i + a_i + b_i\right)}
          \right),
          \label{eqn:p1_beta_binomial}
      \end{align}
      so $Y_i \sim \operatorname{BetaBinomial}\left(
        N_i, a_i, b_i\right)$.

      Using Equation \ref{eqn:p1_beta_binomial}, the expectation of $Y_i$ is
      \begin{equation}
        \mathbb{E}\left[Y_i\right]
        = \sum_{y=0}^{N_i}y\mathbb{P}\left(
        Y_i = y
          \right)
          = \sum_{y=1}^{N_i}y\mathbb{P}\left(
          Y_i = y
        \right).
        \label{eqn:p1_expectation_definition}
      \end{equation}

      Note that when $N_i = 1$, Equation \ref{eqn:p1_expectation_definition}
      trivially becomes $a_i/\left(a_i + b_i\right)$. In general, we can show
      that $\displaystyle\mathbb{E}\left[Y_i\right] = N_i\frac{a_i}{a_i +
        b_i}$. With the $N_i = 1$ base case established, we now have
      % \tiny{
        \begin{align}
          \mathbb{E}\left[Y_i\right]
        &= \sum_{y=1}^{N_i}y\mathbb{P}\left(Y_i = y\right) \nonumber\\
        &= \sum_{y=1}^{N_i} y {N_i \choose y}
          \left(
          \frac{\Gamma\left(a_i + b_i\right)}
          {\Gamma\left(a_i\right)\Gamma\left(b_i\right)}
          \right)
          \left(
          \frac{\Gamma\left(y + a_i\right)\Gamma\left(
          N_i - y + b_i
          \right)}
          {\Gamma\left(N_i + a_i + b_i\right)}
          \right) \nonumber\\
          &= \frac{N_i}{N_i - 1 + a_i + b_i} \sum_{y=1}^{N_i}
            \left(y - 1 + a_i\right)
            \operatorname{BetaBinomial}_{N_i - 1,a_i,b_i}
            \left(y - 1\right) \nonumber\\
        &= \frac{N_i}{N_i - 1 + a_i + b_i}
          \left(
          \frac{\left(N_i - 1\right)a_i}{a_i + b_i} + a_i
          \right) = N_i\frac{a_i}{a_i + b_i} 
          \label{eqn:p1_expectation_derivation}
        \end{align}
      % }
        \normalsize as expected. Substituting $a_i = dp_i$ and
        $b_i = d\left(1-p_i\right)$, we have that
      \begin{equation}
        \mathbb{E}\left[
          Y_i \right]
        = N_i\frac{dp_i}{dp_i + d\left(1 - p_i\right)}  = N_ip_i.
        \label{eqn:p1_expectation_substitution}
      \end{equation}
      
      For the variance, we can use the law of total variance to obtain
      \begin{align}
        \operatorname{var}\left(
        Y_i
        \right)
        &= \mathbb{E}\left[
          \operatorname{var}\left(Y_i \mid q_i\right)
          \right] +
          \operatorname{var}\left(\mathbb{E}\left[Y_i \mid q_i\right]\right)
          \nonumber\\
        &= \mathbb{E}\left[N_i q_i\left(1 - q_i\right)\right]
          + \operatorname{var}\left(N_iq_i\right) \nonumber\\
        &= N_i\left(
          \frac{a_i}{a_i + b_i}
          - \left(
          \frac{a_ib_i}{\left(a_i + b_i\right)^2\left(a_i + b_i + 1\right)}
          +  \left(\frac{a_i}{a_i + b_i}\right)^2
          \right)
          \right) \nonumber\\
        &~~~+ N_i^2\frac{a_ib_i}{\left(a_i + b_i\right)^2\left(a_i + b_i + 1\right)}
          \nonumber\\
        &= N_i\frac{a_ib_i\left(a_i+b_i+ N_i\right)}{
          \left(a_i + b_i\right)^2\left(a_i + b_i + 1\right)}.
          \label{eqn:p1_total_variance}
      \end{align}

      From Equations \ref{eqn:p1_expectation_derivation} and
      \ref{eqn:p1_total_variance}, we obtain the second moment
      \begin{align*}
        \mathbb{E}\left[Y_i^2\right]
        &= \operatorname{var}\left(Y_i\right) +
          \left(\mathbb{E}\left[Y_i\right]\right)^2 \\
        &= N_i\frac{a_ib_i\left(a_i+b_i+ N_i\right)}{
          \left(a_i + b_i\right)^2\left(a_i + b_i + 1\right)}
          + \left(N_i\frac{a_i}{a_i + b_i}\right)^2 \\
        &= N_i\frac{a_i\left(N_i\left(a_i + 1\right) + b_i\right)}{
          \left(a_i + b_i\right)\left(a_i + b_i + 1\right)}.
      \end{align*}

      Substituting $a_i = dp_i$ and $b_i = d\left(1-p_i\right)$ into Equation
      \ref{eqn:p1_total_variance}, we have
      \begin{align}
        \operatorname{var}\left(Y_i\right)
        &= N_ip_i\left(1 - p_i\right)\frac{a_i + b_i + N_i}{a_i + b_i + 1}
          \nonumber\\
        &= N_ip_i\left(1 - p_i\right)\frac{a_i + b_i + 1 + \left(N_i - 1\right)}
          {a_i + b_i + 1} \nonumber\\
        &= N_ip_i\left(1 - p_i\right) \times \left[
          1 + \left(N_i - 1\right)\frac{1}{d + 1}\right].
          \label{eqn:p1_variance_substitution}
      \end{align}

      Thus, we have that $\boxed{\tau^2 = 1/\left(d + 1\right)}$, so small
      values of $d$ mean that the $Z_{ij}$ are highly correlated. This is
      consistent with the behavior of the beta distribution since for small $d$,
      $q_i$ is likely to be close to $0$ and $1$.
    \end{description}
\end{enumerate}
\item In this question a simulation study to investigate the impact on inference
  of omitting covariates in logistic regression will be performed, in the
  situation in which the covariates are independent of the exposure of
  interest. Let $x$ be the covariate of interest and $z$ another
  covariate. Suppose the true (adjusted) model is independently distributed
  $Y_i \mid x_i,z_i \sim \operatorname{Bernoulli}\left(p_i\right)$, where
  \begin{equation}
    \log\frac{p_i}{1 - p_i} = \beta_0 + \beta_1x_i + \beta_2z_i.
    \label{eqn:p2_pi}.
  \end{equation}
  A comparison with the unadjusted model
  $Y_i \mid x_i \sim \operatorname{Bernoulli}\left(p_i^\star\right)$
  independently distributed, where
  \begin{equation}
    \log\frac{p_i^\star}{1 - p_i^\star} = \beta_0^\star + \beta_1^\star x_i,
    \label{eqn:p2_pi_star}
  \end{equation}

  for $i = 1,\ldots,n=1000$ will be made. Suppose $x$ is binary with
  $\mathbb{P}\left(X = 1\right) = 0.5$ and $Z \sim \mathcal{N}\left(0, 1\right)$
  independent and identically distributed with $x$ and $z$
  independent. Combinations of the parameters
  $\beta_1 \in \left\{ 0.5, 1 \right\}$ and
  $\beta_2 \in \left\{0.5,1.0,2.0,3.0\right\}$, with $\beta_0 = -2$ in all cases
  will be considered.

  For each combination of parameters compare the results from the two models,
  Equation \ref{eqn:p2_pi} and Equation \ref{eqn:p2_pi_star}, with respect to:
  \begin{enumerate}
  \item $\mathbb{E}\left[\hat{\beta}_1\right]$ and
    $\mathbb{E}\left[\hat{\beta}_1^\star\right]$, as compared to $\beta_1$.
  \item The standard errors of $\hat{\beta}_1$ and $\hat{\beta}_1^\star$.
  \item The coverage of 95\% confidence intervals for $\beta_1$ and
    $\beta_1^\star$.
  \item The probability of rejecting $H_0 : \beta_1 = 0$ (the power) under both
    models using a Wald test.
  \end{enumerate}

  Based on the results, summarize the effect of omitting a covariate that is
  independent of the exposure of interest, in particular in comparison with the
  linear model.

  \begin{table}
    \scriptsize
    \centering
    \begin{tabular}{llrrrrr}
\toprule
   &        &       Mean &  Standard deviation &     Median &  5\% quantile &  95\% quantile \\
Length (mm) & Parameter &            &                     &            &              &               \\
\midrule
1  & $\alpha$ &   3.304722 &            0.113267 &   3.303521 &     3.122608 &      3.490909 \\
   & $\eta$ &   8.945404 &            1.989280 &   8.837034 &     5.846150 &     12.363148 \\
10 & $\alpha$ &   2.369973 &            0.045122 &   2.369507 &     2.298008 &      2.443862 \\
   & $\eta$ &  15.779253 &            3.575655 &  15.611898 &    10.309790 &     21.820534 \\
20 & $\alpha$ &   1.849104 &            0.069687 &   1.847223 &     1.738584 &      1.965827 \\
   & $\eta$ &   8.176279 &            1.934859 &   8.019274 &     5.281827 &     11.535180 \\
50 & $\alpha$ &   1.732334 &            0.047075 &   1.731810 &     1.656464 &      1.810707 \\
   & $\eta$ &  11.302895 &            2.636021 &  11.133819 &     7.279310 &     15.768627 \\
\bottomrule
\end{tabular}

    \caption{Result of doing $2^{20}$ simulations for each set of parameters.}
    \label{tab:p2_summary}
  \end{table}
  
  \begin{description}
  \item[Solution:] See Table \ref{tab:p2_summary}. For each set of parameters
    $2^{20}$ simulations were done. The expectation was estimated with Monte
    Carlo integration. The standard error was estimated by taking the square root
    of the unbiased variance estimate.
    
    Using the standard error estimate, 95\% confidence intervals were calculated
    for each simulation to see if they contained the true parameter value.

    The estimated standard error was also used to compute the Wald test
    statistic. To estimate the probability of rejecting the null hypothesis
    ($\beta_1 = 0$), a 95\% test was performed for each simulation.

    Estimates for $\beta_1$ are biased in both the adjusted and unadjusted
    model. In the adjusted model, the bias is very small and does not vary much
    with $\beta_2$. In the unadjusted model, $\hat\beta_1^\star$ has bias
    depedent on $\beta_2$.

    When $\beta_2$ is small, say $0.5$, the bias is not too bad, and the 95\%
    confidence interval contains the true parameter value close to 95\% of the
    time as we would expect. As $\beta_2$ grows larger, the estimate becomes
    more and more biased, coverage of the confidence interval worsens, and the
    power of the Wald test declines. When $\beta_1 = 1.0$, the power is still
    high because the effect size is large.

    In the adjusted model, regardless of $\beta_2$, we obtain the expected 95\%
    coverage of the confidence interval, and the power stays consistently high.

    One way to understand this behavior is by looking at the score function:
    \begin{equation}
      S\left(\beta\right) = X^\intercal\left(
        Y - \frac{1}{1 + \exp\left(-X\beta\right)}
      \right).
      \label{eqn:p2_score}
    \end{equation}

    Note that \[
      0 = \mathbb{E}\left[S\left(\beta\right) \mid \beta \right]
      = X^\intercal\left(\mathbf{p} - \frac{1}{1 + \exp\left(-X\beta\right)}\right),
    \]
    where
    $\mathbf{p} = \begin{pmatrix} p_1 & p_2 & \cdots &
      p_n \end{pmatrix}^\intercal$. Thus, we are choosing $\beta$ to fit
    $\operatorname{expit}\left(X\beta\right)$ to $\mathbf{p}$.  In the adjusted
    model, this amounts to choosing
    \begin{align}
      \hat{\beta}_0^\star
      &= \operatorname{logit}\left(
      \frac{\sum_{i=1}^n \left(1 - x_i\right)y_i}{\sum_{i=1}^n\left(1 - x_i\right)}
        \right) =
        \operatorname{logit}\hat{\mathbb{P}}\left(Y_i = 1\mid X_i = 0\right)
        \label{eqn:p2_beta_0_star}\\
      \hat{\beta}_0^\star + \hat{\beta}_1^\star
      &=
      \operatorname{logit}\left(
      \frac{\sum_{i=1}^n x_iy_i}{\sum_{i=1}^n 1 - x_i}
      \right) =
        \operatorname{logit}\hat{\mathbb{P}}\left(Y_i = 1\mid X_i = 1\right).
        \label{eqn:p2_beta_1_star}
    \end{align}

    Let $\phi$ be the probability density function for the standard normal
    distribution. Due to nonlinearity, we have that
    \begin{align*}
      \mathbb{P}\left(Y_i = 1\mid X_i = 0\right)
      &= \int_{-\infty}^\infty \operatorname{expit}\left(
        \beta_0 + \beta_2z
        \right)\phi\left(z\right)\,\mathrm{d}z 
        \neq \operatorname{expit}\left(\beta_0\right)\\
      \mathbb{P}\left(Y_i = 1\mid X_i = 1\right)
      &= \int_{-\infty}^\infty \operatorname{expit}\left(
        \beta_0 + \beta_1 + \beta_2z
        \right)\phi\left(z\right)\,\mathrm{d}z 
        \neq \operatorname{expit}\left(\beta_0 + \beta_1\right).
    \end{align*}    
    Substituting the true values for the observed values in Equations
    \ref{eqn:p2_beta_0_star} and \ref{eqn:p2_beta_1_star}, it's clear that we're
    fitting to the incorrect log odds ratio.

    Code for the simulations can be found in
    \href{http://nbviewer.jupyter.org/github/ppham27/stat570/blob/master/hw8/logistic\_regression.ipynb}{\texttt{logistic\_regression.ipynb}}.
  \end{description}
\end{enumerate}
\end{document}
