\documentclass[letterpaper,11pt]{article}

\usepackage{amsmath}
\usepackage{amssymb}
\usepackage[hmargin=1.25in,vmargin=1in]{geometry}
\usepackage{booktabs}
\usepackage{graphicx}
\usepackage{hyperref}
\usepackage{lmodern}
\usepackage{microtype}
\usepackage{pdflscape}
\usepackage{subcaption}

\title{Coursework 5: STAT 570}
\author{Philip Pham}
\date{\today}

\begin{document}
\maketitle

\begin{enumerate}
\item Consider the data given in Table \ref{tab:p1_data}, which are a simplified
  version of those reported in Breslow and Day (1980). These data arose from a
  case-control study that was carried out to investigate the relationship
  between esophageal cancer and various risk factors. Disease status is denoted
  $Y$ with $Y = 0$ and $Y = 1$ corresponding to without/with disease and alcohol
  consumption is represented by $X$ with $X = 0$ and $X = 1$ denoting less than
  80g and greater than or equal to 80g on average per day. Let the probabilities
  of high alcohol consumption in the cases and controls be denoted

  \begin{equation}
    p_1 = \mathbb{P}\left(X = 1 \mid Y = 1\right)~\text{and}~
    p_2 = \mathbb{P}\left(X = 1 \mid Y = 0\right),
  \end{equation}
  respectively.
  
  \begin{table}
    \centering
    \begin{tabular}{l|cc|c}
      & $X = 0$ & $X = 1$ & \\
      \midrule
    \end{tabular}
    \caption{Case-control data: $Y = 1$ corresponds to the event of esophageal
      cancer, and $X = 1$ exposure to greater than 80g of alcohol per day. There
      are 200 cases and 775 controls.}
    \label{tab:p1_data}
  \end{table}
  
\end{enumerate}
\end{document}
