\documentclass[letterpaper,11pt]{article}

\usepackage{amsmath}
\usepackage{amssymb}
\usepackage[hmargin=1.25in,vmargin=1in]{geometry}
\usepackage{booktabs}
\usepackage{graphicx}
\usepackage{hyperref}
\usepackage{lmodern}
\usepackage{microtype}
\usepackage{pdflscape}
\usepackage{subcaption}

\title{Coursework 5: STAT 570}
\author{Philip Pham}
\date{\today}

\begin{document}
\maketitle

\begin{enumerate}
\item Consider the data given in Table \ref{tab:p1_data}, which are a simplified
  version of those reported in Breslow and Day (1980). These data arose from a
  case-control study that was carried out to investigate the relationship
  between esophageal cancer and various risk factors. Disease status is denoted
  $Y$ with $Y = 0$ and $Y = 1$ corresponding to without/with disease and alcohol
  consumption is represented by $X$ with $X = 0$ and $X = 1$ denoting less than
  80g and greater than or equal to 80g on average per day. Let the probabilities
  of high alcohol consumption in the cases and controls be denoted

  \begin{equation}
    p_1 = \mathbb{P}\left(X = 1 \mid Y = 1\right)~\text{and}~
    p_2 = \mathbb{P}\left(X = 1 \mid Y = 0\right),
    \label{eqn:p1_pi_definition}
  \end{equation}
  respectively. Further, let $X_1$ be the number exposed from $n_1$ cases and
  $X_2$ be the number exposed from $n_2$ controls. Suppose
  $X_i \mid p_i \sim \operatorname{Binomial}(n_i,p_i)$ in the case ($i = 1$) and
  control ($i = 2$) groups.
  
  \begin{table}
    \centering
    \begin{tabular}{l|cc|c}
      & $X = 0$ & $X = 1$ & \\
      \hline
      $Y = 1$ & 104 & 96 & 200 \\
      $Y = 0$ & 666 & 109 & 775 \\
    \end{tabular}
    \caption{Case-control data: $Y = 1$ corresponds to the event of esophageal
      cancer, and $X = 1$ exposure to greater than 80g of alcohol per day. There
      are 200 cases and 775 controls.}
    \label{tab:p1_data}
  \end{table}

  \begin{enumerate}
  \item Of particular interest in studies such as this is the odds ratio defined
    by
    \begin{equation}
      \theta = \frac{\mathbb{P}\left(Y = 1 \mid X = 1\right)/\mathbb{P}\left(Y = 0 \mid X = 1\right)}
      {\mathbb{P}\left(Y = 1 \mid X = 0\right)/\mathbb{P}\left(Y = 0 \mid X = 0\right)}.
      \label{eqn:p1_odds_ratio}
    \end{equation}

    Show that the odds ratio is equal to
    \begin{equation}
      \theta
      = \frac{\mathbb{P}\left(X = 1 \mid Y = 1\right)/\mathbb{P}\left(X = 0 \mid Y = 1\right)}
      {\mathbb{P}\left(X = 1 \mid Y = 0\right)/\mathbb{P}\left(X = 0 \mid Y = 0\right)}
      = \frac{p_1/(1 - p_1)}{p_2/(1-p_2)}.
      \label{eqn:p1_odds_ratio_solution}
    \end{equation}

    \begin{description}
    \item[Solution:] We have that
      \begin{equation}
        \mathbb{P}\left(Y = y \mid X = x\right)
        = \frac{\mathbb{P}\left(X = x \mid Y = y\right)\mathbb{P}\left(Y = y\right)}{\mathbb{P}\left(
            X = x
          \right)}
        \label{eqn:p1_bayes_rule}
      \end{equation}
      by Bayes' rule. Applying Equation \ref{eqn:p1_bayes_rule} to Equation
      \ref{eqn:p1_odds_ratio}, we get
      \begin{equation}
        \theta = \frac{
          \left[
            \mathbb{P}\left(X = 1 \mid Y = 1\right)\mathbb{P}\left(Y = 1\right)
          \right]/\left[
            \mathbb{P}\left(X = 0 \mid Y = 1\right)\mathbb{P}\left(Y = 0\right)
          \right]}{
          \left[
            \mathbb{P}\left(X = 0 \mid Y = 1\right)\mathbb{P}\left(Y = 1\right)
          \right]/\left[
            \mathbb{P}\left(X = 0 \mid Y = 0\right)\mathbb{P}\left(Y = 0\right)
          \right]}.
      \end{equation}
      The $\mathbb{P}\left(Y = y\right)$ factors cancel and we obtain the first
      part of Equation \ref{eqn:p1_odds_ratio_solution}. Using Equation
      \ref{eqn:p1_pi_definition}, we substitute to obtain the second part of
      Equation \ref{eqn:p1_odds_ratio_solution}.
    \end{description}
  \item Obtain the MLE and a 90\% confidence interval for $\theta$, for the data
    of Table \ref{tab:p1_data}.

    \begin{description}
    \item[Solution:]
      The likelihood and log-likelihood functions are
      \begin{align}
        L\left(p_1,p_2\right)
        &= {n_1 \choose x_1}p_1^{x_1}\left(1 - p_1\right)^{n_1 - x_1} +
          {n_2 \choose x_2}p_2^{x_2}\left(1 - p_2\right)^{n_2 - x_2}
          \label{eqn:p1_likelihood}\\
        l\left(p_1,p_2\right)
        &= \log L\left(p_1,p_2\right) \nonumber\\
        &= \sum_{i=1}^2\left[
          \log {n_i \choose x_i} + x_i \log p_i + \left(n_i - x_i\right)\log\left(1 - p_i\right)
          \right],
        \nonumber
      \end{align}
      so the score function is
      \begin{equation}
        S\left(p_1,p_2\right)
        = \nabla \log L\left(p_1,p_2\right)
        = \begin{pmatrix}
          \frac{x_1 - n_1p_1}{p_1\left(1 - p_1\right)} \\
          \frac{x_2 - n_2p_2}{p_2\left(1 - p_2\right)}.
        \end{pmatrix}
        \label{eqn:p1_score}
      \end{equation}
      
      Thus, the Fisher information is
      \begin{equation}
        I\left(p_1,p_2\right) = mathbb{E}\left[
          S\left(p_1,p_2\right)S\left(p_1,p_2\right)^\intercal
        \right]= \begin{pmatrix}
          \frac{n_1}{p_1\left(1 - p_1\right)} & 0 \\
          0 & \frac{n_2}{p_2\left(1 - p_2\right)}
        \end{pmatrix}.
        \label{eqn:p1_fisher_information}
      \end{equation}

      From Equation \ref{eqn:p1_score}, we can solve
      $S\left(\hat{p}_1,\hat{p}_2\right) = \mathbf{0}$ to get the MLEs
      $\hat{p}_1 = x_1/n_1$ and $\hat{p}_2 = x_2/n_2$. Since the MLE is
      invariant to reparameterization, we have the MLE for $\theta$:
      \begin{equation}
        \boxed{\hat{\theta} = \frac{\hat{p}_1/\left(1 - \hat{p}_1\right)}{\hat{p}_2/\left(1 - \hat{p}_2\right)} = \frac{1992}{1417} \approx 5.640.}
      \end{equation}

      We estimate the confidence interval for $\log\hat{\theta}$ which works
      since $\log$ is a monotonic transform. Using the delta method and Equation
      \ref{eqn:p1_fisher_information}, we have that
      \begin{align}
        \operatorname{Var}\left(
        \log\hat{\theta}\right)
        &\approx \left(\nabla \log\hat{\theta}\right)^\intercal
        \left(I\left(\hat{p}_1,\hat{p}_2\right)\right)^{-1}
          \left(\nabla \log\hat{\theta}\right) \nonumber\\
        &=
          \begin{pmatrix}
            \frac{1}{\hat{p}_1\left(1 - \hat{p}_1\right)} &
            \frac{1}{\hat{p}_2\left(1 - \hat{p}_2\right)}
          \end{pmatrix}
          \begin{pmatrix}
            \frac{\hat{p}_1\left(1 - \hat{p}_1\right)}{n_1} & 0 \\
            0 & \frac{\hat{p}_2\left(1 - \hat{p}_2\right)}{n_2}
          \end{pmatrix}
                \begin{pmatrix}
                  \frac{1}{\hat{p}_1\left(1 - \hat{p}_1\right)} \\
                  \frac{1}{\hat{p}_2\left(1 - \hat{p}_2\right)}
                \end{pmatrix} \nonumber\\
        &= \frac{1}{n_1\hat{p}_1\left(1 - \hat{p}_1\right)} +
          \frac{1}{n_2\hat{p}_2\left(1 - \hat{p}_2\right)} \nonumber\\
        &= \frac{1}{n_1\hat{p}_1} + \frac{1}{n_1\left(1 - \hat{p}_1\right)} +
          \frac{1}{n_2\hat{p}_2} + \frac{1}{n_2\left(1 - \hat{p}_2\right)}.
      \end{align}
      Numerically, this is
      $\operatorname{Var}\left(\log\hat{\theta}\right) \approx 0.0307$.

      The 90\% confidence interval for $\log\hat{\theta}$ is approximately
      \begin{equation}
        \left(
          \log\hat{\theta} -
          \Phi^{-1}\left(0.95\right)
          \sqrt{\operatorname{Var}\left(\log\hat{\theta}\right)},
          \log\hat{\theta} +
          \Phi^{-1}\left(0.95\right)
          \sqrt{\operatorname{Var}\left(\log\hat{\theta}\right)}
        \right),
      \end{equation}
      which is about $\left(1.441, 2.018\right)$. Taking the exponent of both
      sides, we have a 90\% confidence interval for $\hat{\theta}$ of
      $\boxed{\left(4.228, 7.524\right).}$
    \end{description}
  \item We now consider a Bayesian analysis. Assume that the prior distribution
    for $p_i$ is the beta distribution $\operatorname{Beta}\left(a, b\right)$
    for $i = 1, 2$. Show that the posterior distribution $p_i \mid x_i$ is given
    by the beta distribution $\operatorname{Beta}\left(a + x_i, b + n_i - x_i\right)$,
    $i=1,2$.
    \begin{description}
    \item[Solution:] From Equation \ref{eqn:p1_likelihood}, we have that the
      posterior:
      \begin{align*}
        p\left(p_i \mid X_i = x_i\right)
        &\propto \mathbb{P}\left(X_i = x_i \mid p_i\right)p\left(p_i\right) \\
        &\propto p_i^{x_i + a - 1}\left(1 - p_i\right)^{n_i - x_i + b - 1}.
      \end{align*}
      Integration from $0$ to $1$, we have the beta fnction, so
      \begin{equation}
        p\left(p_i \mid X_i = x_i\right)
        = \frac{\Gamma\left(a + x_i + b + n_i - x_i\right)}{\Gamma\left(a + x_i\right)\Gamma\left(b + n_i - x_i\right)}p_i^{a + x_i - 1}\left(1 - p_i\right)^{b + n_i - x_i - 1},
      \end{equation}
      which is the $\operatorname{Beta}\left(a + x_i, b + n_i - x_i\right)$
      distribution.
    \end{description}
  \item Consider the case $a = b = 1$. Obtain expressions for the posterior
    mean, mode, and standard deviation. Evaluate these posterior summaries for
    the data of Table \ref{tab:p1_data}. Report 90\% posterior credible
    intervals for $p_1$ and $p_2$.

    \begin{description}
    \item[Solution:] For $a = b = 1$, we have that
      $p_1 \mid x_1 \sim \operatorname{Beta}\left(97, 105\right)$ and
      $p_2 \mid x_2 \sim \operatorname{Beta}\left(110, 667\right)$.

      For the posterior means, we have that
      $\mathbb{E}\left[p_1 \mid x_1\right] = 97/202$ and
      $\mathbb{E}\left[p_2 \mid x_2\right] = 110/777$.

      The mode of a $\operatorname{Beta}\left(\alpha,\beta\right)$ distributed
      random variable is $\frac{\alpha - 1}{\alpha + \beta - 2}$. So, for the
      posterior modes, we have that
      $\operatorname{mode}\left(p_1 \mid x_1\right) = 12/25$ and
      $\operatorname{mode}\left(p_2 \mid x_2\right) = 109/775$.

      The variance of a $\operatorname{Beta}\left(\alpha,\beta\right)$
      distributed random variable is
      $\frac{\alpha\beta}{(\alpha+\beta)^2(\alpha + \beta + 1)}$. For
      $p_1 \mid x_1$ and $p_2 \mid x_2$, we have standard errors:      
      \begin{align*}
        \sigma_{p_1 \mid x_1}
        &= \frac{1}{202}\sqrt{\frac{10185}{203}} \approx 0.0351 \\
        \sigma_{p_2 \mid x_2}
        &= \frac{1}{777}\sqrt{\frac{36685}{389}} \approx 0.0125.
      \end{align*}

      For the 90\% credible interval, I choose $l$ and $u$ such that
      $\mathbb{P}\left(\left[l, u\right]\right) = 0.9$,
      $\mathbb{P}\left(\left(-\infty, l\right)\right) = 0.05$ and
      $\mathbb{P}\left(\left(u, \infty\right)\right) = 0.05$. This is called the
      \emph{equal-tailed interval}.

      For $p_1 \mid x_1$, the interval is $\left[0.4226, 0.5380\right]$. For
      $p_2 \mid x_2$, the inverval is $\left[0.1215, 0.1626\right]$ This is
      computed numerically with \texttt{scipy.stats.beta.interval} in
      \href{https://nbviewer.jupyter.org/github/ppham27/stat570/blob/master/hw5/case\_control.ipynb}{\texttt{case\_control.ipynb}}.
    \end{description}
  \item Obtain the asymptotic form of the posterior distribution and obtain 90\%
    credible intervals for $p_1$ and $p_2$. Compare this interval with the exact
    calculation of the previous part.

    \begin{description}
    \item[Solution:] 
    \end{description}
  \end{enumerate}
\item
  \begin{enumerate}
  \item Consider the likelihood,
    $\hat\theta \mid \theta \sim \mathcal{N}\left(\theta,V\right)$ and the prior
    $\theta \sim \mathcal{N}\left(0,W\right)$ with $V$ and $W$ known. Show that
    $\theta \mid \hat{\theta} \sim \mathcal{N}\left(r\hat{\theta},rV\right)$, where
    $r=W/\left(V +W\right)$.
    \begin{description}
    \item[Solution:] This result follows from the conjugacy of the normal
      distribution with itself:
      \begin{align}
        p\left(\theta \mid \hat{\theta}\right)
        &\propto p\left(\hat{\theta}\mid \theta \right)p\left(\theta\right) \nonumber\\
        &\propto \exp\left(
          -\frac{1}{2V}\left(\hat{\theta} - \theta\right)^2
          -\frac{1}{2W}\theta^2
          \right) \nonumber\\
        &\propto \exp\left(
          -\frac{V + W}{2\left(VW\right)}
          \left(\frac{W}{V+W}\hat{\theta}^2 -2\frac{W}{V+W}\hat\theta\theta + \theta^2\right)
          \right) \nonumber\\
        &\propto \exp\left(
          -\frac{V + W}{2\left(VW\right)}
          \left(\theta - \frac{W}{V+W}\hat{\theta}\right)^2
          \right) = \exp\left(
          -\frac{1}{2\left(rV\right)}
          \left(\theta - r\hat{\theta}\right)^2
          \right) \nonumber
      \end{align}
      after completing the square. We recognize this distribution as being part
      of the normal family, which gives us the result.
    \end{description}
  \item Suppose we wish to compare the models $M_0$: $\theta = 0$ versus $M_1$:
    $\theta \neq 0$. Show that the Bayes factor is given by
    \begin{equation}
      \frac{p\left(\hat\theta \mid M_0\right)}{p\left(\hat\theta \mid M_1\right)}
      = \frac{1}{\sqrt{1-r}}\exp\left(-\frac{Z^2}{2}r\right),
      \label{eqn:p2_bayes_factor}
    \end{equation}
    where $Z = \hat{\theta}/\sqrt{V}$.

    \begin{description}
    \item[Solution:] We have that
      \begin{align*}
        p\left(\hat{\theta} \mid M_0\right)
        &= p\left(\hat{\theta} \mid \theta = 0\right)
          = \frac{1}{\sqrt{2\pi V}}\exp\left(-\frac{1}{2V}\hat{\theta}^2\right) \\
        p\left(\hat{\theta} \mid M_1\right)
        &= \int_{-\infty}^\infty p\left(\hat{\theta} \mid \theta\right)p\left(\theta\right)\,\mathrm{d}\theta \\
        &= \frac{1}{\sqrt{2\pi\left(V + W\right)}}\exp\left(
          -\frac{1}{2\left(V + W\right)}\hat\theta^2
          \right)
      \end{align*}
      after completing the square. Substituting into the left-hand side of
      Equation \ref{eqn:p2_bayes_factor}, we obtain
      \begin{equation*}
        \frac{p\left(\hat\theta \mid M_0\right)}{p\left(\hat\theta \mid M_1\right)}
        = \sqrt{\frac{V+W}{V}}\exp\left(-\frac{1}{2}\cdot \frac{W}{V+W} \cdot \frac{\hat\theta^2}{V}\right)
        = \frac{1}{\sqrt{1-r}}\exp\left(-\frac{Z^2}{2}r\right)
      \end{equation*}
      as desired. 
    \end{description}
  \item Suppose we have a prior probability $\pi_1 = \mathbb{P}\left(M_1\right)$
    of model $M_1$ being true. Write down an expression for the posterior
    probability $\mathbb{P}\left(M_1 \mid \hat{\theta}\right)$ in terms of the
    Bayes factor.
    \begin{description}
    \item[Solution:] Let $K$ be the Bayes factor. By applying Bayes' rule, we
      have that
      \begin{align*}
        \mathbb{P}\left(
          M_1 \mid \hat\theta
        \right)
        &=
        \frac
        {\mathbb{P}\left(\hat\theta \mid M_1\right)\mathbb{P}\left(M_1\right)}
        {\mathbb{P}\left(\hat\theta \mid M_0\right)\mathbb{P}\left(M_0\right) +
          \mathbb{P}\left(\hat\theta \mid M_1\right)\mathbb{P}\left(M_1\right)} \\
        &= \frac
          {K^{-1}\mathbb{P}\left(\hat\theta \mid M_0\right)\pi_1}
        {\mathbb{P}\left(\hat\theta \mid M_0\right)\left(1 - \pi_1\right) +
          K^{-1}\mathbb{P}\left(\hat\theta \mid M_0\right)\pi_1} \\
        &= \frac
          {K^{-1}\pi_1}
          {\left(1 - \pi_1\right) + K^{-1}\pi_1}
          = \frac
          {\pi_1}{K\left(1 - \pi_1\right) + \pi_1}.
      \end{align*}
    \end{description}
  \item Now suppose we have summaries from two studies, $\theta_j$, $V_j$,
    $j = 1,2$. Assuming,
    $\theta_j \mid \theta \sim \mathcal{N}\left(\theta, V_j\right)$ and the
    prior $\theta \sim \mathcal{N}\left(0,W\right)$, derive the posterior
    $p\left(\theta \mid \theta_1,\theta_2\right)$.
    \begin{description}
    \item[Solution:] We have
      \begin{align*}
        p\left(\theta \mid \theta_1,\theta_2\right)
        &\propto
          p\left(\theta_2 \mid \theta_1, \theta \right)
          p\left(\theta_1 \mid \theta \right)p\left(\theta\right)
          =
          p\left(\theta_2 \mid \theta \right)
          p\left(\theta_1 \mid \theta \right)p\left(\theta\right) \\
        &\propto
          \exp\left(
          -\frac{1}{2V_2}
          \left(\theta_2 - \theta\right)^2
          \right)
          \exp\left(
          -\frac{V_1 + W}{2\left(V_1W\right)}
          \left(\theta - \frac{W}{V_1+W}\theta_1\right)^2
          \right) \\
        &\propto
          \exp\left(
          -\frac{V_1V_2 + V_1W + V_2W}{2\left(V_1V_2W\right)}
          \left(
          \theta - \frac{V_2W\theta_1 + V_1W\theta_2}{V_1V_2 + V_1W + V_2W}
          \right)^2
          \right),
      \end{align*}
      after repeatedly completing the square and dropping factors that don't
      depend on $\theta$.

      Thus, we have that
      \begin{equation}
        \theta \mid \theta_1,\theta_2 \sim \mathcal{N}\left(
          \frac{V_2W\theta_1 + V_1W\theta_2}{V_1V_2 + V_1W + V_2W},
          \frac{V_1V_2W}{V_1V_2 + V_1W + V_2W}
        \right).
      \end{equation}
    \end{description}
  \item Derive the Bayes factor
    \begin{equation}
      \frac{p\left(\theta_1,\theta_2\mid M_0\right)}
      {p\left(\theta_1,\theta_2\mid M_1\right)},
    \end{equation}
    again comparing the models $M_0$: $\theta = 0$ versus $M_1$:
    $\theta \neq 0$.

    \begin{description}
    \item[Solution:] $\left(\theta_1,\theta_2\right)$ have a bivariate normal
      distribution. Under $M_0$, we have that
      \begin{align}
        p\left(\theta_1,\theta_2 \mid M_0\right)
        &= p\left(\theta_1,\theta_2 \mid \theta = 0\right) \nonumber\\
        &= \frac{1}{2\pi\sqrt{V_1V_2}}\exp\left(
          -\frac{1}{2}
          \begin{pmatrix}
            \theta_1 & \theta_2
          \end{pmatrix}
                       \begin{pmatrix}
                         \frac{1}{V_1} & 0 \\
                         0 & \frac{1}{V_2}
                       \end{pmatrix}
                       \begin{pmatrix}
                         \theta_1 \\ \theta_2
                       \end{pmatrix}
          \right).
      \end{align}
      
      Under $M_1$, we have that
      \begin{align}
        p\left(\theta_1,\theta_2 \mid M_1\right)
        &= \int_{-\infty}^\infty p\left(\theta_1,\theta_2 \mid \theta\right)
          p\left(\theta\right)\,\mathrm{d}\theta.
      \end{align}
      We can consider $\theta$ as having the improper prior
      $\mathcal{N}\left(\mathbf{0}, \begin{pmatrix} W & W \\ W &
          W \end{pmatrix}\right),$ which results in
      \begin{equation}
        \theta_1,\theta_2 \mid M_1 \sim \mathcal{N}\left(
          \mathbf{0}, \begin{pmatrix}
            V_1 + W & W \\
            W & V_2 + W
          \end{pmatrix}          
        \right)
      \end{equation}
      by conjugacy of the multivariate normal distribution.

      The Bayes factor can then be computed:
      \begin{equation}
        \sqrt{\frac{V_1V_2 + V_1W + V_2W}{V_1V_2}}
        \exp\left(
          -\frac{1}{2}
          \begin{pmatrix}
            \theta_1 & \theta_2
          \end{pmatrix}
          \Lambda
          \begin{pmatrix}
            \theta_1 \\ \theta_2
          \end{pmatrix}
        \right),
      \end{equation}
      where \begin{equation}
        \Lambda =
        \begin{pmatrix}
          \frac{1}{V_1} & 0 \\
          0 & \frac{1}{V_2}
        \end{pmatrix}
        + \frac{1}{V_1V_2 + V_1W + V_2W}
        \begin{pmatrix}
          V_2 + W & -W \\
          -W & V_1 + W
        \end{pmatrix}.
      \end{equation}
    \end{description}
  \item We will show these results can be used in the context of a genome-wide
    association study on Type II diabetes, reported by Frayling et al. (2007,
    Science). Two sets of data were independently collected, resulting in two
    log odds ratios $\hat{\theta}_j$, $j = 1,2$, for each SNP.
     
    For SNP rs9939609 point estimates (95\% confidence intervals) were 1.27
    (1.16, 1.37) and 1.15 (1.09,1.23). Suppose we have a normal prior for the
    odds ratio that has a 95\% range (0.67, 1.50).

    \begin{enumerate}
    \item Find W from this interval, and then calculate the posterior median and
      95\% intervals for $\theta$ based on (i) the first dataset only, (ii) both
      of the populations.
      \begin{description}
      \item[Solution:] 
      \end{description}
      
    \item 
    \item 
    \end{enumerate}
  \end{enumerate}
\end{enumerate}
\end{document}
