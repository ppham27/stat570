\documentclass[letterpaper,11pt]{article}

\usepackage{amsmath}
\usepackage{amssymb}
\usepackage[hmargin=1.25in,vmargin=1in]{geometry}
\usepackage{booktabs}
\usepackage{graphicx}
\usepackage{hyperref}
\usepackage{lmodern}
\usepackage{microtype}
\usepackage{subcaption}

\title{Coursework 2: STAT 570}
\author{Philip Pham}
\date{\today}

\begin{document}
\maketitle
\begin{enumerate}
\item Consider the simple linear regression model
  \begin{equation*}
    Y_i = \beta_0 + \beta_1x_i + \epsilon_i,~i = 1,\ldots,n,
    \label{eqn:p1_model}    
  \end{equation*}
  where the error terms $\epsilon_i$ are such that
  $\mathbb{E}\left[\epsilon_i\right] = 0$,
  $\operatorname{Var}\left(\epsilon_i\right) = \sigma^2$, and
  $\operatorname{Cov}\left(\epsilon_i, \epsilon_j\right) = 0$ for $i \neq j$.

  In the following you will consider
  $x_i \sim_{\mathrm{iid}} \mathcal{N}\left(20, 3^2\right)$, with $\beta_0 = 2$
  and $\beta_1 = -2.5$ and $n=15,30$.

  Consider the model in Equation \ref{eqn:p1_model} with the error terms
  $\epsilon_i$, independent and identically distributed, from the distributions:
  \begin{itemize}
  \item The normal distribution with mean $0$ and variance $2^2$.
  \item The uniform distribution on the range $(-r,r)$ for $r = 2$.
  \item A skew normal distribution with $\alpha = 5$, $\omega = 1$, and $\xi$
    chosen to given mean $0$.
  \end{itemize}

  \begin{enumerate}
  \item What is the theoretical bias for $\hat{\beta}$ if the errors are of the
    form specified?

    \begin{description}
    \item[Solution:] The theoretical bias for $\hat{\beta}$ is $0$. Let
      \begin{equation}
        X = \begin{pmatrix}
          1 & x_1 \\
          1 & x_2 \\
          \vdots & \vdots \\
          1 & x_n
        \end{pmatrix}
        \label{eqn:p1_X}
      \end{equation}
      
      If we use the least squares estimate, we have
      \begin{align}
        \hat{\beta}
        &= \left(X^\intercal X\right)^{-1}X^\intercal y \nonumber\\
        &= \left(X^\intercal X\right)^{-1}X^\intercal\left(X\beta + \epsilon \right) \nonumber\\
        &= \beta + \left(X^\intercal X\right)^{-1}X^\intercal\epsilon,
          \label{eqn:p1_beta_hat}
      \end{align}

      Thus, using Equation \ref{eqn:p1_beta_hat} and linearity of expectations,
      we have
      \begin{equation}
        \boxed{
          \operatorname{bias}\left(\hat{\beta}\right) =
          \mathbb{E}\left[\hat{\beta}\right] - \beta =
          \beta +
          \left(X^\intercal X\right)^{-1}X^\intercal
          \mathbb{E}\left[\epsilon\right] - \beta
          = 0.
        }
        \label{eqn:p1_beta_hat_bias}
      \end{equation}
    \end{description}
  \item Compare the variance of the estimator as reported by least squares, with
    that which follows from the sampling distribution of the estimator.

    \begin{description}
    \item[Solution:] The variances reported by least squares are in the column
      titled \textbf{Least-squares Variance} in Tables
      \ref{tab:p1_simulations_beta_0} and \ref{tab:p1_simulations_beta_1}. This
      number is the variance reported by least squares averaged across the
      simulations.

      To obtain the sample variance, 2,048 simulations were done. The results
      are reported in the column titled \textbf{Sample Variance} in Tables
      \ref{tab:p1_simulations_beta_0} and \ref{tab:p1_simulations_beta_1}.

      The two variance estimates are nearly identical except when the errors
      have a $t$ distribution.

      Code for calculations can be found in
      \href{https://nbviewer.jupyter.org/github/ppham27/stat570/blob/master/hw2/estimator\_variance.ipynb}{\texttt{estimator\_variance.ipynb}}.
    \end{description}
    
  \item Examine the distribution of the resultant estimators (across
    simulations) of $\beta_0$ and $\beta_1$, in particular with respect to
    normality. For each parameter find the coverage probability of a 95\%
    confidence interval, that is the proportion of times that the confidence
    intervals contain the true value.

    \begin{description}
    \item[Solution:] Several tests of normality were done. First, I checked how
      frequently the 95\% confidence interval contains the true value of
      $\beta_j$. In the second-to-last column of Tables
      \ref{tab:p1_simulations_beta_0} and \ref{tab:p1_simulations_beta_1}, this
      was always close to 95\% as one would expect.

      I also performed the
      \href{https://en.wikipedia.org/wiki/Shapiro\%E2\%80\%93Wilk\_test}{Shapiro-Wilk
        test}, whose null hypothesis is that the estimates were drawn from a
      normal distribution. From the last column of Tables
      \ref{tab:p1_simulations_beta_0} and \ref{tab:p1_simulations_beta_1}, we
      can clearly reject the null hypothesis when the errors come from a
      $t$-distribution. For the other error distributions, the evidence is not
      as conclusive.

      Finally, I did a qualitative evaluation by plotting the histogram against
      a fitted normal distribution. In Figures \ref{fig:p1_beta_0_distribution}
      and \ref{fig:p1_beta_1_distribution}, we see the normal distribution
      accurately describes the data except when the errors are $t$-distributed.
      
      Code for calculations and plots can be found in
      \href{https://nbviewer.jupyter.org/github/ppham27/stat570/blob/master/hw2/estimator\_variance.ipynb}{\texttt{estimator\_variance.ipynb}}.
    \end{description}
    
  \item \textbf{Bonus:} Can you ``break'' least squares? i.e., find a
    distribution of the errors (with mean zero) that produces poor confidence
    interval coverage?
    
    \begin{description}
    \item[Solution:] Yes, it is possible to ``break'' least squares. The
      \href{https://en.wikipedia.org/wiki/Gauss\%E2\%80\%93Markov\_theorem}{Gauss-Markov
        theorem} gives us the conditions under which the least squares estimate
      is the best linear unbiased estimator: (1) the errors have mean 0, (2)
      they are homoscedastic, and (3) they are uncorrelated.

      The $t$-distribution with $3/2$ degrees has infinite variance. If our
      errors are distributed in such a manner, from Tables
      \ref{tab:p1_simulations_beta_0} and \ref{tab:p1_simulations_beta_1} and
      Figures \ref{fig:p1_beta_0_distribution} and
      \ref{fig:p1_beta_1_distribution}, our estimates no longer have a normal
      distribution.

      In particular, we see that least squares poorly estimates the variance,
      and the estimates have the greatest error from the true values of
      $\beta_j$. Despite the non-normality and badly estimated variances,
      confidence interval coverage is actually quite good.
    \end{description}
  \end{enumerate}

  \begin{table}
    \tiny
    \begin{subtable}{\textwidth}
      \centering
      \begin{tabular}{rlrrrrr}
\toprule
 $n$ & Error Distribution &  $\hat{\beta}_0$ Estimate &  Sample Variance &  Least-squares Variance &  95\% CI Coverage &  Shapiro-Wilk $p$-value \\
\midrule
  15 &             normal &                  1.981029 &        16.066709 &               16.462905 &         0.954102 &                0.440320 \\
  15 &            uniform &                  2.108807 &         5.488093 &                5.457236 &         0.948730 &                0.052002 \\
  15 &        skew normal &                  2.013222 &         1.706046 &                1.624730 &         0.943359 &                0.558366 \\
  15 &                  t &                  2.223518 &       268.857783 &              447.847807 &         0.950195 &                0.000000 \\
  30 &             normal &                  2.023642 &         6.058565 &                6.067812 &         0.949707 &                0.635323 \\
  30 &            uniform &                  1.980790 &         2.039209 &                2.001988 &         0.941895 &                0.307742 \\
  30 &        skew normal &                  2.021499 &         0.566532 &                0.591055 &         0.954590 &                0.099939 \\
  30 &                  t &                  1.468913 &       981.048353 &              559.866048 &         0.950684 &                0.000000 \\
\bottomrule
\end{tabular}

      \caption{Simulations of for $\hat{\beta}_0$. Recall that $\beta_0 = 2$.}
      \label{tab:p1_simulations_beta_0}
    \end{subtable}
    \begin{subtable}{\textwidth}
      \centering
      \begin{tabular}{rlrrrrr}
\toprule
 $n$ & Error Distribution &  $\hat{\beta}_1$ Estimate &  Sample Variance &  Least-squares Variance &  95\% CI Coverage &  Shapiro-Wilk $p$-value \\
\midrule
  15 &             normal &                 -2.499933 &         0.037590 &                0.038154 &         0.953613 &                0.447169 \\
  15 &            uniform &                 -2.505455 &         0.012717 &                0.012647 &         0.948242 &                0.045104 \\
  15 &        skew normal &                 -2.500507 &         0.003944 &                0.003765 &         0.944336 &                0.907067 \\
  15 &                  t &                 -2.512479 &         0.543700 &                1.037909 &         0.948730 &                0.000000 \\
  30 &             normal &                 -2.502034 &         0.015370 &                0.015198 &         0.949219 &                0.744967 \\
  30 &            uniform &                 -2.499262 &         0.005189 &                0.005014 &         0.936523 &                0.101658 \\
  30 &        skew normal &                 -2.500797 &         0.001421 &                0.001480 &         0.956055 &                0.850093 \\
  30 &                  t &                 -2.475815 &         2.229763 &                1.402287 &         0.950684 &                0.000000 \\
\bottomrule
\end{tabular}

      \caption{Simulations of for $\hat{\beta}_1$. Recall that
        $\beta_1 = -2.5$.}
      \label{tab:p1_simulations_beta_1}
    \end{subtable}
    \caption{Results for different sample sizes and error distributions.}
    \label{tab:p1_simulations}
  \end{table}

  \begin{figure}
    \centering
    \includegraphics{p1_beta_hat_0_distribution.pdf}
    \caption{The distributions of $\hat{\beta}_0$ compared to a fitted normal.}
    \label{fig:p1_beta_0_distribution}
  \end{figure}

  \begin{figure}
    \centering
    \includegraphics{p1_beta_hat_1_distribution.pdf}
    \caption{The distributions of $\hat{\beta}_1$ compared to a fitted normal.}
    \label{fig:p1_beta_1_distribution}
  \end{figure}
  
\item Consider the exponential regression problem with independent responses
  \begin{equation}
    p\left(y_i \mid \lambda_i\right) = \lambda_i\exp\left(-\lambda_iy_i\right),
    y_i > 0,
    \label{eqn:p2_model}
  \end{equation}
  and $\log\lambda_i = \beta_0 + \beta_1x_i$ for given covariates $x_i$,
  $i = 1,\ldots,n$. We wish to estimate the $2 \times 1$ regression parameter
  $\beta = \begin{pmatrix}
    \beta_0 & \beta_1
  \end{pmatrix}^\intercal$ using maximum likelihood estimation (MLE).

  \begin{enumerate}
  \item Find expressions for the likelihood function $L\left(\beta\right)$,
    log-likelihood function $l\left(\beta\right)$, score function
    $S\left(\beta\right)$, and Fisher's information matrix $I\left(\beta\right)$.

    \begin{description}
    \item[Solution:] We can rewrite Equation \ref{eqn:p2_model} in terms of
      $\beta$, which gives us
      \begin{align}
        p\left(y_i \mid \beta_0,\beta_1\right)
        &= \exp\left(\beta_0 + \beta_1x_i\right)
        \exp\left(- y_i\exp\left(\beta_0 + \beta_1x_i\right)\right) \nonumber\\
        &=
        \exp\left(
          \beta_0 + \beta_1x_i - y_i\exp\left(\beta_0 + \beta_1x_i\right)\right)
        \label{eqn:p2_model_beta}.
      \end{align}

      Using Equation \ref{eqn:p2_model_beta}, we can write the likelihood
      function
      \begin{align}
        L\left(\beta\right)
        &= \prod_{i=1}^n p\left(y_i \mid \beta_0,\beta_1\right) \nonumber\\
        &= \exp\left(
          n\beta_0 + \beta_1\sum_{i=1}^n x_i
          -
          \sum_{i=1}^n y_i\exp\left(\beta_0 + \beta_1x_i\right)
          \right).
          \label{eqn:p2_likelihood}
      \end{align}

      Taking the log of Equation \ref{eqn:p2_likelihood}, we have the
      log-likelihood function as
      \begin{align}
        l\left(\beta\right)
        &= \log L\left(\beta\right) \nonumber\\
        &= n\beta_0 + \beta_1\sum_{i=1}^nx_i -
          \sum_{i=1}^ny_i\exp\left(\beta_0 + \beta_1x_i\right).
          \label{eqn:p2_log_likelihood}
      \end{align}

      Taking the gradient of Equation \ref{eqn:p2_log_likelihood}, we have the
      score function
      \begin{align}
        S\left(\beta\right)
        &= \nabla l\left(\beta\right) \nonumber\\
        &= \begin{pmatrix}
          n - \sum_{i=1}^ny_i\exp\left(\beta_0 + \beta_1x_i\right) \\
          \sum_{i=1}^nx_i - \sum_{i=1}^n x_iy_i\exp\left(\beta_0 + \beta_1x_i\right)
        \end{pmatrix}.
        \label{eqn:p2_score}
      \end{align}

      One definition of the Fisher information is the expected value of the
      observed information which is the negative of the second derivative of the
      log-likelihood function. For a single observation,
      \begin{align}
        \mathcal{I}_1\left(\beta\right) &= \mathbb{E}\left[          
          \begin{pmatrix}
            Y\exp\left(\beta_0 + \beta_1x_i\right)
            & x_iY\exp\left(\beta_0 + \beta_1x_i\right) \\
            x_iY\exp\left(\beta_0 + \beta_1x_i\right) &
            x_i^2Y\exp\left(\beta_0 + \beta_1x_i\right)
          \end{pmatrix} \mid X = x_i\right] \nonumber\\
        &= \frac{1}{\exp\left(\beta_0 + \beta_1x_i\right)} \begin{pmatrix}
          \exp\left(\beta_0 + \beta_1x_i\right)
          & x_i\exp\left(\beta_0 + \beta_1x_i\right) \\
          x_i\exp\left(\beta_0 + \beta_1x_i\right) &
          x_i^2\exp\left(\beta_0 + \beta_1x_i\right)
        \end{pmatrix} \nonumber\\
        &= \begin{pmatrix}
          1 & x_i \\
          x_i & x_i^2
        \end{pmatrix}
        \label{eqn:p2_fisher_information_single}      
      \end{align}
      by properties of the exponential distribution. Thus, Fisher information is
      \begin{equation}
        \mathcal{I}_n\left(\beta\right) =
        \begin{pmatrix}
          n & \sum_{i=1}^n x_i \\
          \sum_{i=1}^n x_i & \sum_{i=1}^n x_i^2
        \end{pmatrix}.
        \label{eqn:p2_fisher_information}
      \end{equation}
    \end{description}
  \item Find expressions for the maximum likelihood estimate $\hat{\beta}$. If
    no closed form solution exists, then instead provide a functional form that
    could be simply implemented for solution.

    \begin{description}
    \item[Solution:] We can solve for $\hat{\beta}_0$ in terms of
      $\hat{\beta}_1$. We know that $S\left(\hat{\beta}\right) = \mathbf{0}$.

      From Equation \ref{eqn:p2_score}, we can solve for $\hat{\beta}_0$,
      \begin{align}
        \hat{\beta}_0
        &= \log n - \log \sum_{i=1}^n y_i\exp\left(\hat{\beta}_1x_i\right).
        \label{eqn:p2_beta_0_hat}
      \end{align}

      Substituing Equation \ref{eqn:p2_beta_0_hat} into the second entry of
      Equation \ref{eqn:p2_score}, we have
      \begin{align}
        0
        &= \sum_{i=1}^n x_i -
          \exp\left(\hat{\beta}_0\right)
          \sum_{i=1}^nx_iy_i\exp\left(\hat{\beta}_1x_i\right) \nonumber\\
        &= \sum_{i=1}^n x_i -
          \frac{n}{\sum_{i=1}^ny_i\exp\left(\hat{\beta}_1x_i\right)}
          \sum_{i=1}^nx_iy_i\exp\left(\hat{\beta}_1x_i\right),
        \label{eqn:p2_beta_1_hat}
      \end{align}
      which we can solve numerically with a root-finding algorithm.            
    \end{description}
  \item For the data in Table \ref{tab:p2_data}, numerically maximize the
    likelihood function to obtain estimates of $\beta$. These data consist of
    the survival times ($y$) of rats as function of concentration of a
    contaminant ($x$). Find the asymptotic covariance matrix for your estimate
    using the information $\mathcal{I}\left(\beta\right)$. Provide a 95\%
    confidence interval for each element of $\beta_0$ and $\beta_1$.
    
    \begin{table}
      \centering
      \begin{tabular}{crr}
        \toprule
        $i$ & $x_i$ & $y_i$ \\
        \midrule
        1 & 6.1 & 0.8 \\
        2 & 4.2 & 3.5 \\
        3 & 0.5 & 12.4 \\
        4 & 8.8 & 1.1 \\
        5 & 1.5 & 8.9 \\
        6 & 9.2 & 2.4 \\
        7 & 8.5 & 0.1 \\
        8 & 8.7 & 0.4 \\
        9 & 6.7 & 3.5 \\
        10 & 6.5 & 8.3 \\
        11 & 6.3 & 2.6 \\
        12 & 6.7 & 1.5 \\
        13 & 0.2 & 16.6 \\
        14 & 8.7 & 0.1 \\
        15 & 7.5 & 1.3 \\
        \bottomrule
      \end{tabular}
      \caption{Each observation is a rat. $x_i$ are the concentrations of the
        contaminant, and $y_i$ are the survival times.}
      \label{tab:p2_data}
    \end{table}

    \begin{description}
    \item[Solution:] Numerically solving Equations \ref{eqn:p2_beta_0_hat} and
      \ref{eqn:p2_beta_1_hat}, we have that
      \begin{equation}
        \hat{\beta} =
        \begin{pmatrix}
          \hat{\beta}_0 \\
          \hat{\beta}_1
        \end{pmatrix}
        =
        \begin{pmatrix}
          -2.821150253077923 \\
          0.30133576292327585
        \end{pmatrix}.
        \label{eqn:p2_beta_hat_mle}
      \end{equation}

      The Fisher information gives a lower bound on the variance according to
      the Cram\'er-Rao bound. Asymptotic normality of the MLE tells us that
      \begin{equation*}
        \hat{\beta}_n - \beta \rightarrow
        \mathcal{N}\left(0, \mathcal{I}_n^{-1}\left(\beta\right)\right)
      \end{equation*}
      in distribution.

      Thus, we have the covariance matrix
      \begin{equation}
        \operatorname{Var}\left(\hat{\beta}\right)
        \approx \begin{pmatrix}
          15 & 90.1 \\
          90.1 & 671.07
        \end{pmatrix}^{-1}
        = \begin{pmatrix}
          0.34448471 & -0.04625162 \\
          -0.04625162 & 0.00770005
        \end{pmatrix}.
      \end{equation}

      Using this we can approximate 95\% confidence intervals as
      $\hat{\beta}_j \pm
      z_{0.975}\sqrt{\operatorname{Var}\left(\hat{\beta}_j\right)}$, where
      $z_{p} = \Phi^{-1}\left(p\right)$ and $\Phi$ is the cumulative
      distribution function of the normal distribution.

      We have the confidence intervals
      \begin{align}
        \left(
        \hat{\beta}_0 - 1.150358,
        \hat{\beta}_0 + 1.150358
        \right)
        &= \left(
          -3.97150839, -1.67079212
        \right) \nonumber\\
        \left(
        \hat{\beta}_1 - 0.171986669,
        \hat{\beta}_1 + 0.171986669
        \right)
        &= \left(
          0.12934909, 0.47332243
          \right)
          \label{eqn:p2_confidence_interval}
      \end{align}
      for $\hat{\beta}_0$ and $\hat{\beta}_1$, respectively.
      

      All calcuations can be found in
      \href{https://nbviewer.jupyter.org/github/ppham27/stat570/blob/master/hw2/exponential\_regression.ipynb}{\texttt{exponential\_regression.ipynb}}.
    \end{description}
  \item Plot the log-likelihood function $l\left(\beta_0,\beta_1\right)$ and
    compare with the log of the asymptotic normal approximation to the sampling
    distribution of the MLE.
    \label{part:2d}

    \begin{description}
    \item[Solution:] The two plots can be found in Figure
      \ref{fig:p2_beta_hat_likelihood}.

      The log-likelihood function can be found in Figure
      \ref{fig:p2_log_likelihood}, and the asymptotic normal approximation is
      plotted in Figure \ref{fig:p2_asymptotic_normal}.

      The distributions are similar, and the asymptotic normal approximation
      accurately captures the covariance structure of $\hat{\beta}$.  The
      asymptotic normal approximation, however, does underestimate the
      variance. This is unsurprising since the inverse of the Fisher information
      matrix is a lower bound on variance according to the Cram\'er-Rao bound.

      Code for the plots can be found in
      \href{https://nbviewer.jupyter.org/github/ppham27/stat570/blob/master/hw2/exponential\_regression.ipynb}{\texttt{exponential\_regression.ipynb}}.
      
      \begin{figure}
        \begin{subfigure}[b]{\textwidth}
          \centering
          \includegraphics{p2_log_likelihood.pdf}
          \caption{The log-likelihood function is centered at the MLE estimate.}
          \label{fig:p2_log_likelihood}
        \end{subfigure}
        \begin{subfigure}[b]{\textwidth}
          \centering
          \includegraphics{p2_asymptotic_normal.pdf}
          \caption{The asymptotic normal approximation mirrors the
            log-likelihood with slightly less variance.}
          \label{fig:p2_asymptotic_normal}
        \end{subfigure}
        \caption{Plots of the distributions of $\hat{\beta}$ for Part
          \ref{part:2d}.}
        \label{fig:p2_beta_hat_likelihood}
      \end{figure}      
    \end{description}
    
  \item Summarize the results of the estimation presented above in a manner that
    would address the question of whether increasing concentrations of the
    contaminant had an effect on a rat's life expectancy.

    \begin{description}
    \item[Solution:] $\lambda_i$ can be viewed as the rate of death, so the mean
      survival time is $\lambda_i^{-1}$. $\log\lambda_i$ has a linear
      relationship with concentrations of the contaminant described by
      $\beta_1$. $\beta_1 > 0$ indicates that increasing concentrations of the
      contaminant increase death rates, and therefore, decrease life expectancy.

      Indeed from Equation \ref{eqn:p2_beta_hat_mle}, our estimate of $\beta_1$,
      $\hat{\beta}_1 = 0.30133576292327585 > 0$. From Equation
      \ref{eqn:p2_confidence_interval}, we see that the 95\% confidence interval
      lies entirely on the positive half-line, so the result is statistically
      significant at level $0.05$.

      Thus, it is quite likely that increasing concentrations of the contaminant
      decrease life expectancy.
    \end{description}
  \end{enumerate}  
\end{enumerate}
\end{document}
